\documentclass[a4paper,12pt]{article}
\usepackage[top = 2.5cm, bottom = 2.5cm, left = 2.5cm, right = 2.5cm]{geometry} 
\usepackage[T1]{fontenc}
\usepackage[utf8]{inputenc}
\usepackage{multirow} % Multirow is for tables with multiple rows within one cell.
\usepackage{booktabs} % For even nicer tables.

% As we usually want to include some plots (.pdf files) we need a package for that.
\usepackage{graphicx} 
\usepackage{amsmath} % to use split function

% The default setting of LaTeX is to indent new paragraphs. This is useful for articles. But not really nice for homework problem sets. The following command sets the indent to 0.
\usepackage{setspace}
\setlength{\parindent}{0in}

\usepackage{CJKutf8} % chinese template

\usepackage{listings}
\usepackage{xcolor}

\definecolor{codegreen}{rgb}{0,0.6,0}
\definecolor{codegray}{rgb}{0.5,0.5,0.5}
\definecolor{codepurple}{rgb}{0.58,0,0.82}
\definecolor{backcolour}{rgb}{0.95,0.95,0.92}

\lstdefinestyle{mystyle}{
    backgroundcolor=\color{backcolour},   
    commentstyle=\color{codegreen},
    keywordstyle=\color{magenta},
    numberstyle=\tiny\color{codegray},
    stringstyle=\color{codepurple},
    basicstyle=\ttfamily\footnotesize,
    breakatwhitespace=false,         
    breaklines=true,                 
    captionpos=b,                    
    keepspaces=true,                 
    numbers=left,                    
    numbersep=5pt,                  
    showspaces=false,                
    showstringspaces=false,
    showtabs=false,                  
    tabsize=2
}

\lstset{style=mystyle}


% Package to place figures where you want them.
\usepackage{float}
\usepackage{algorithm} 
\usepackage{algpseudocode} 
% The fancyhdr package let's us create nice headers.
\usepackage{fancyhdr}
\pagestyle{fancy} % With this command we can customize the header style.

\fancyhf{} % This makes sure we do not have other information in our header or footer.

\lhead{\footnotesize ITM Homework 1}% \lhead puts text in the top left corner. \footnotesize sets our font to a smaller size.

%\rhead works just like \lhead (you can also use \chead)
\rhead{\footnotesize Yu-Chieh Kuo} %<---- Fill in your lastnames.

% Similar commands work for the footer (\lfoot, \cfoot and \rfoot).
% We want to put our page number in the center.
\cfoot{\footnotesize \thepage} 

\begin{document}
\thispagestyle{plain} % This command disables the header on the first page. 

\begin{tabular}{p{15.5cm}} % This is a simple tabular environment to align your text nicely 
{\large \bf Introduction of Text Mining} \\
National Taiwan University, Fall 2020  \\
\hline % \hline produces horizontal lines.
\\
\end{tabular} % Our tabular environment ends here.

\vspace*{0.3cm} % Now we want to add some vertical space in between the line and our title.

\begin{center} % Everything within the center environment is centered.
	{\Large \bf Homework 1} % <---- Don't forget to put in the right number
	\vspace{2mm}
	
        % YOUR NAMES GO HERE
	{\bf Yu-Chieh Kuo B07611039} % <---- Fill in your names here!
		
\end{center}  
\vspace{0.4cm}
\begin{CJK*}{UTF8}{bsmi}

\section{Overview} 這份作業要求我們使用python package求出給定輸入資料的TF-IDF vectors,並計算前兩份TF-IDF vectors之間的cosine similarity。我針對給定的輸入文件進行一些前處理,包括轉成小寫、去除英文的stop words,並簡單去除掉標點符號後利用sklearn套件內的函數求出給定資料的TFIDF vector,並求出前兩個vector之間的cosine similarity約為0.200。

\section{Code Explanation}
以下是我的原始程式碼,此份程式碼也會包在壓縮檔內一同繳交。
\begin{lstlisting}[language=Python, caption=Python code]
from nltk.tokenize import word_tokenize
from collections import Counter
from sklearn.feature_extraction.text import CountVectorizer
from sklearn.feature_extraction.text import TfidfVectorizer
from sklearn.metrics.pairwise import cosine_similarity
import numpy as np
import os
import nltk

all_file = list(range(1, 1096))

for i in range(1, 1096):
    ### read file
    doc_name = "~/NTUCourse/NTU-IM-ITM/HW-01/PA1-data/" + str(i) + ".txt"
    file = os.path.expanduser(doc_name)
    f = open(file)
    docs = f.read()
    #print(docs)
    
    ### lowerize
    token_lower = docs.lower()
    #print(token_lower)
    
    ### stop punctuation
    token_sequence = word_tokenize(token_lower)
    stop_punc = [',',';','.','\'','?']
    tokens_wo_stop_punc = [x for x in token_sequence if x not in stop_punc]
    #print(tokens_wo_stop_punc)
    
    ### stop words in english
    stop_words = nltk.corpus.stopwords.words('english')
    tokens_wo_stop_words = [x for x in tokens_wo_stop_punc if x not in stop_words]
    #print(tokens_wo_stop_words)
    
    lst = ' '.join([str(elem) for elem in tokens_wo_stop_words])
    final_docs = []
    final_docs.append(lst)
    #print(final_docs)
    
    all_file[i-1] = lst
    
### calculate TFIDF vectors
TFIDF_vectorizer = TfidfVectorizer()
TFIDF_vectors = TFIDF_vectorizer.fit_transform(all_file)

### outfile the 1.vec and 2.vec
for i in range(1,3):
    filename = "doc" + str(i) + ".txt"
    outF = open(filename, "w")
    for line in TFIDF_vectors.toarray()[i - 1]:
        outF.write(str(line))
        outF.write("\n")
    outF.close()

### print the cosine similarity
print("The cosine similarity of 1.vec and 2.vec is",cosine_similarity(TFIDF_vectors[0], TFIDF_vectors[1]).flatten()[0])
\end{lstlisting}
以下將講解我如何處理資料,以及如何求出TFIDF vectors與cosine similarity。
\begin{enumerate}
\item { \textbf{Import packages:}
讀入本次作業需要的套件,如nltk、sklearn、numpy、os等,示於第一行至第八行,其中os用於讀檔時需指定檔案路徑,其餘套件則是如上課所教。}

\item { \textbf{Read files: }
第十四至第十七行是用來將檔案讀入python中,由於讀檔需要透過絕對路徑找尋檔案,因此需要用到os這個套件。我透過迴圈將檔案存進doc這個變數中。
}

\item { \textbf{Text pre-processing:}
第二十一至第三十七行則是對文字進行前處理,我將文件中所有英文字轉成小寫,並去除標點符號與英文中的停止單字。值得一提的是透過套件內的方程式所產生出來的文字會是很多element的list,所以我於第三十五至三十七行將這些切割好的單字重新連成一個list,並將這些處理好的文字assign給預先定義好的list中,以利後續計算TFIDF vectors。
}

\item { \textbf{TFIDF vectors:}
第四十三、四十四行就是透過上課所教的方法計算TFIDF vectors。
}

\item { \textbf{Output files:}
第四十七行至第五十三行是將TFIDF vector輸出成兩份文字檔1.vec與2.vec,流程就只是讀$TFIDF \_ vectors$ 這個變數中的內容,並將他寫入欲輸出的檔案中。
}

\item { \textbf{Cosine similarity:}
第五十六行就是印出前兩個vector之間的cosine similarity,也是透過上課教的方法,用套件內的函數直接計算兩份vector之間的cosine similarity。最後算出來的結果約為0.200。
}
\end{enumerate}


\end{CJK*}
\end{document}
