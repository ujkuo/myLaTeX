\documentclass{OR_SolutionTheme}

\title{Operations Research, Spring 2022 (110-2) \\
Suggestions for Common Issues in Homework}

\begin{document}
\maketitle

In homework 1, we (TAs) found couple of common issues varying from notations usage
to formulation logic. Thus, we list these common issues to help you 
feel more comfortable and obtain higher grades 
in arriving homework, case assignments, and most importantly, the final exam and the term project.

The content of this suggestion document is fourfold.
First, we want to show you the benefits and 
the skills of parametrizing and variable expression
in \cref{sec:notation}.
In addition, we list several typical formatting issues 
in your homework 1 and provide better ways 
to conquer these issues in \cref{sec:format}.
Moreover, in \cref{sec:misc} we list two
miscellaneous but essential matters you may read 
before starting your homework 2. 
Last but not least, to incentive you to dive into \LaTeX, 
we deliver a few selected example topics about \LaTeX \
in \cref{sec:latex}.

\section{Notation Usage}\label{sec:notation}
\begin{description}
    \item[Parameters and variables:]
    Parametrizing the problem is the initial step to formulate tasks,
    and properly defining and naming sets, indices, parameters, and variables
    are helpful for readability for sure.
    Typically, we use the capital letter to denote the parameters, and
    use the lower case letter to denote the decision variables.
    
    \item[Properly define set:]
    Parametrizing may also significantly ease your work. For example,
    in problem 1, tens of students define a few sets and parameters and use the exact
    number to indicate which player can play as a setter or an opposite. 
    It's fine if typing all of them correctly and whenever the problem is not complicated.
    Nonetheless, we see lots of typos happening in the case of using an exact number as an index.
    Establishing a parameter or a set to demonstrate your idea
    is more flexible and scalable despite typos,. You may be rescued from plenty of typing by 
    a well-defined precedent parameter or set, and it's more efficient if you want or 
    need to extend your formulations.
    
    \item[Set expression:]
    In problem 1, we see some students try to
    use numbers to represent positions, and they
    write sentences such as
    \[
        \begin{array}{c}
            \text{Let } J \text{ be the set of position, }
            J=\Set{1,2,3,4,5}(S,L,H,M,O),    \\
            \text{Let } i\in I \text{ be the 
            index of position, where } 
            I=\Set{1(S),2(L),3(H),4(M),5(O)}.
        \end{array}
    \]
    We may understand what you want to express,
    but it will confuse your set definition.
    It's better to explain your idea
    in additional words or sentences.
    
    \item[Define what you exactly need:]
    We see several students defining redundant
    or dependent parameters and variables, and
    never using them or messily using them in the formulation.
    We don't deduct this issue this time, but
    you should check what you write before submitting.

\end{description}

\section{Formatting issues}\label{sec:format}
\begin{description}
    \item[Incorrect symbol:]
    Many students use mathematical symbols pretty professional, but still a few have some 
    problems. For example, the symbol representing less equal is $\leq$, not $<=$. 
    Several documents, guidance, and introductions on the Internet teach about mathematical writing,
    and you can search for them easily.\footnote{For example, I (one of the TAs) usually reference
    \href{https://www.caam.rice.edu/~heinken/latex/symbols.pdf}{this document}.}
    
    \item[Operator and verb:]
    We see several students writing 
    \[
        \text{Let } x_{ij} = \text{number of something,}
    \]
    or quite similar sentences. 
    It would help if you did not use a mathematical symbol as a verb.
    In contrast, you can write
    \[
        \begin{array}{C}
            \text{Let } x_{ij} \text{ be the number of something,} \\
            \text{Denote the number of something by } x_{ij}, 
        \end{array}
    \]
    or other equivalent sentences.
    
    \item[Italics or regular:]
    This is a common error, and plenty of students lose their grades
    by this issue.
    For example, in problem 3, if you want to 
    represent a shift $i$ belonging to the afternoon type in a parameter, 
    says $B_{i,\mathrm{afternoon}}$, you need to specify the type afternoon by
    a regular font, instead of an {\it italics}. The same argument is applicable to
    the definition of set. You can see how the 
    instructor writes when defining a set $S=\Set{\mathrm{morning, afternoon, night, leave}}$.
    
    In contrast, if you want to type
    your variables, parameters, or sets in the
    text, use italics.
    
    \item[Punctuation:]
    When you finish your formulation, please don't forget
    to type a comma or a period ('.', not '。') in the end.
    
\end{description}

\section{Miscellaneous issues}\label{sec:misc}
\begin{description}
    \item[Course name:] This course is called {\bf Operation\textcolor{red}{s} Research},
    instead of Operation Research, Operation Researchs, Operations Researchs, or
    Operating research.
    
    \item[\LaTeX:]
    We strongly encourage you to type your work in \LaTeX.
    For those who have a weak preference to install \LaTeX \ locally,
    you can use \href{https://www.overleaf.com/}{Overleaf}, an
    online \LaTeX \ editor which is easy to use without installation
    and environment setting. Overleaf also supports real-time collaboration and version control, and its gallery contains
    hundreds of \LaTeX \ templates. Using \LaTeX \ can easily 
    make your work professional in mathematical writing, 
    and save you a lot of time.
    
\end{description}
\section{\LaTeX \ Tips}\label{sec:latex}
In this section, we summarize some basic \LaTeX \ syntax
and tips which may be useful and save your time.
\begin{description}
    \item[Frequent syntax:]
    You can search lots of \LaTeX \ resources and 
    detailed documents online. 
    This section only demonstrates some useful skills 
    you may use in the following assignments.
    \begin{enumerate}
        \item We take the solution for problem 2 in homework 1 for example.
        \[
            \begin{split}
                \min \quad & \sum_{k\in K}x_k \\
                \mbox{s.t.} \quad & \left(\sum_{k\in K}\sum_{m=1}^4\sum_{n=m+1}^5
                A_{mnk}\right)\sum_{k\in K}x_k \geq D_{ij}
                \quad \forall i\in I, \ j\in J \\
                & x_k \geq 0 \quad \forall k\in K.
            \end{split}
        \]
        To format this IP formulation, you may use the following two methods.
        A tip is that you can auto-adjust the size of 
        the brackets by adding 
        \texttt{\textbackslash left} and
        \texttt{\textbackslash right} 
        before your left and right brackets,
        respectively.
        \begin{minted}{latex}
        \[
            \begin{split}
                \min \quad & \sum_{k\in K}x_k \\
                \mbox{s.t.} \quad & \left(\sum_{k\in K}\sum_{m=1}^4\sum_{n=m+1}^5
                A_{mnk}\right)\sum_{k\in K}x_k \geq D_{ij}
                \quad \forall i\in I, \ j\in J \\
                & x_k \geq 0 \quad \forall k\in K.
            \end{split}
        \]
        \end{minted}
        \begin{minted}{latex}
        \[
            \begin{array}{RLL}
                \min & \sum_{k\in K}x_k & \\
                \mbox{s.t.} & \left(\sum_{k\in K}\sum_{m=1}^4\sum_{n=m+1}^5
                A_{mnk}\right)\sum_{k\in K}x_k \geq
                D_{ij} & \forall i\in I, \ j\in J \\
                & x_k \geq 0 \quad \forall k\in K. &
            \end{array}
        \]
        \end{minted}
        Note that you have to invoke \texttt{amsmath} 
        and \texttt{array}
        packages before writing equations, respectively.
        You can read the \href{https://www.overleaf.com/learn/latex/Aligning_equations_with_amsmath}{\texttt{amsmath}} and 
        \href{https://www.overleaf.com/learn/latex/Tables}{\texttt{array}} package
        details to extend your program.
        
        \item To draw a matrix, for example,
        \[
            \begin{bmatrix}
                1 & 2^x & 3y\\
                \frac{3}{4} & f(q) & c
            \end{bmatrix},
        \]
        you can use the method like
        \begin{minted}{latex}
        \[
            \begin{bmatrix}
                1 & 2^x & 3y\\
                \frac{3}{4} & f(q) & c
            \end{bmatrix}
        \]
        \end{minted}
        and check the \href{https://www.overleaf.com/learn/latex/Matrices}{document} for further use.
        
        \item 
        In homework 2, you need to implement a
        branch-and-bound algorithm and depict 
        the full branch-and-bound tree. 
        You can follow the steps to draw a tree in \LaTeX.
        Note that this method requires a \texttt{tikz} 
        package and set 
        \texttt{\textbackslash newdimen\textbackslash nodeDist}
        and \texttt{\textbackslash nodeDist=35mm}
        (or another preferred distance)
        before
        begining your document, i.e., before
        \texttt{\textbackslash begin\{document\}}.
        Here I draw an example from the previous course slide.
        \newpage
        \begin{figure}
        \centering
        \begin{tikzpicture}[
         node/.style={
          draw,rectangle,
          },]
            \node [node,text width=2cm, align=left] (A) {$x^1=\left(\frac{11}{2},0\right)$ \\[2mm] $z_1=\frac{33}{4}$};
            \path (A) ++(-135:\nodeDist) node [node,text width=2cm, align=left] (B) {$x^2=\left(2,\frac{3}{2}\right)$ \\[2mm] $z_2=\frac{15}{2}$};
            \path (A) ++(-45:\nodeDist) node [node,text width=2cm, align=center] (C) {Infeasible};
            \path (B) ++(-135:\nodeDist) node [node,text width=2cm, align=left] (D) {$x^3=\left(2,1\right)$ \\[2mm] $z_3=7$};
            \path (B) ++(-45:\nodeDist) node [node,text width=2cm, align=left] (E) {$x^4=\left(\frac{7}{4},2\right)$ \\[2mm] $z_4=\frac{29}{4}$};
            \draw (A) -- (B) node [left,pos=0.5] {$x_1\leq2$}(A);
            \draw (A) -- (C) node [right,pos=0.5] {$x_1\geq3$}(A);
            \draw (B) -- (D) node [left,pos=0.5] {$x_2\leq1$}(A);
            \draw (B) -- (E) node [right,pos=0.5] {$x_2\geq2$}(A);
    
        \end{tikzpicture}
        \caption{An example of a branch-and-bound tree}
        \end{figure}
        \begin{minted}{latex}
        \begin{figure}
        \centering
        \begin{tikzpicture}[
         node/.style={
          draw,rectangle,
          },]
            \node [node,text width=2cm, align=left] (A) {$x^1=\left(\frac{11}{2},0\right)$ \\[2mm] $z_1=\frac{33}{4}$};
            \path (A) ++(-135:\nodeDist) node [node,text width=2cm, align=left] (B) {$x^2=\left(2,\frac{3}{2}\right)$ \\[2mm] $z_2=\frac{15}{2}$};
            \path (A) ++(-45:\nodeDist) node [node,text width=2cm, align=center] (C) {Infeasible};
            \path (B) ++(-135:\nodeDist) node [node,text width=2cm, align=left] (D) {$x^3=\left(2,1\right)$ \\[2mm] $z_3=7$};
            \path (B) ++(-45:\nodeDist) node [node,text width=2cm, align=left] (E) {$x^4=\left(\frac{7}{4},2\right)$ \\[2mm] $z_4=\frac{29}{4}$};
            \draw (A) -- (B) node [left,pos=0.5] {$x_1\leq2$}(A);
            \draw (A) -- (C) node [right,pos=0.5] {$x_1\geq3$}(A);
            \draw (B) -- (D) node [left,pos=0.5] {$x_2\leq1$}(A);
            \draw (B) -- (E) node [right,pos=0.5] {$x_2\geq2$}(A);
        \end{tikzpicture}
        \caption{An example of a branch-and-bound tree}
        \end{figure}
        \end{minted}
        
        \item 
        We have learned the simplex method in the
        course, and we are available to draw 
        tableaus of linear programming as well.
        Here I also take the previous course slide
        for example.
        \[
            \begin{array}{ccc}
                \begin{array}{cccc|c}
                     -2 & -3 & 0 & 0 & 0 \\
                     \hline
                     1 & 2 & 1 & 0 & x_3=6 \\
                     \fbox{2} & 1 & 0 & 1 & x_4=8
                \end{array}
                & \rightarrow &
                \begin{array}{cccc|c}
                     0 & -2 & 0 & 1 & 8 \\
                     \hline
                     1 & \fbox{$\frac{3}{2}$} & 1 & -\frac{1}{2} & x_3=2 \\
                     1 & \frac{1}{2} & 0 & \frac{1}{2} & x_1=4
                \end{array}
            \end{array}
        \]
        \newpage
        
        \begin{minted}{latex}
        \[
            \begin{array}{ccc}
                \begin{array}{cccc|c}
                     -2 & -3 & 0 & 0 & 0 \\
                     \hline
                     1 & 2 & 1 & 0 & x_3=6 \\
                     \fbox{2} & 1 & 0 & 1 & x_4=8
                \end{array}
                & \rightarrow &
                \begin{array}{cccc|c}
                     0 & -2 & 0 & 1 & 8 \\
                     \hline
                     1 & \fbox{$\frac{3}{2}$} & 1 & -\frac{1}{2} & x_3=2 \\
                     1 & \frac{1}{2} & 0 & \frac{1}{2} & x_1=4
                \end{array}
            \end{array}
        \]
        \end{minted}
    \end{enumerate}

    % \item[Chinese:]
    % We know some people have a strong preference to answer in Chinese. 
    % You can invoke several packages to type Chinese words in \LaTeX.
    % Here I would like to suggest you use
    % \href{https://ctan.org/pkg/xecjk?lang=en}{\texttt{xeCJK}}. 
    % The only thing you need to do is to invoke \texttt{xeCJK} package 
    % (i.e., \texttt{\textbackslash usepackage\{xeCJK\}}), 
    % and make sure you use \texttt{xeLaTeX} to compile your file. 
    % For Overleaf users, you can check the menu at the left top 
    % and change a compiler in settings.
    % You can specify your font in \texttt{xeCJK} environment
    % by 
    % \texttt{\textbackslash setCJKmainfont\{font name\}} 
    % command,
    % and I (one of TAs) usually use \texttt{Noto Serif CJK TC}.
    
    % 我們知道有些學生對用中文寫作業有著強烈的偏好,
    % 你可以透過使用\texttt{xeCJK}這個package來完成這件事!
    
    
\end{description}

\end{document}