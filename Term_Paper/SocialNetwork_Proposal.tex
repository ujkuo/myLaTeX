\documentclass[a4paper]{article}
%The main tex begins from line 262
\usepackage[top = 2.5cm, bottom = 2.5cm, left = 2cm, right = 2cm]{geometry} 
\usepackage{amsmath} % For the usage of \because and \therefore
\usepackage{amssymb}
\usepackage{fancyhdr}
\usepackage{lastpage}
\usepackage{etoolbox}
\usepackage{indentfirst}
\usepackage{tabularx}
\usepackage{mathtools}
\usepackage{booktabs}
\usepackage{authblk}
\usepackage[calcwidth]{titlesec}
\usepackage{bookmark}
\usepackage[capitalize]{cleveref}
\usepackage{array}
\usepackage[english]{babel}
\usepackage{enumitem} % For the usage of labeling in enumerate and itemize
%\usepackage[utf8]{inputenc}
%\usepackage{CJKutf8}
%\usepackage{xeCJK}

\usepackage{xcolor}
\usepackage{natbib}
\bibliographystyle{econ}
\usepackage{hyperref}
\hypersetup{
        colorlinks,
        citecolor=myblue,
        linkcolor=myblue,
        urlcolor=myred,
}

\usepackage[mathscr]{eucal}
\usepackage{tcolorbox}
%\usepackage[sc]{mathpazo}
%\linespread{1.05}
\usepackage[T1]{fontenc}
%\usepackage{newpxtext,newpxmath}
\usepackage{pxfonts}
%%%%%%%%%%%%%%%%%%%%%%%%%%%%%%%%%%%%%%%%%%%%%%%%%%%

\makeatletter
% Macros for setting basic info
\gdef\@uni{National Taiwan University}
\gdef\@department{Information Management}
\gdef\@me{Yu-Chieh Kuo}

\newcommand{\course}[2][]{
    \ifstrempty{#1}{
        \gdef\shortcourse{#2}
    }{
        \gdef\shortcourse{#1}
    }
    \gdef\@course{#2}
}
\newcommand{\teacher}[1]{\gdef\@teacher{#1}}
\newcommand{\semester}[1]{\gdef\@semester{#1}}
\newcommand{\uni}[1]{\gdef\@uni{#1}}
\newcommand{\department}[1]{\gdef\@department{#1}}
\newcommand{\assignment}[2][Homework]{
    \ifstrempty{#2}{
        \gdef\@assignment{#1}
    }{
        \gdef\@assignment{#1 #2}
    }
}

% page style for title page
\fancypagestyle{title}{
    \fancyhf{}
    \renewcommand{\headrulewidth}{0pt}
    \cfoot{\footnotesize Page {\thepage} of \pageref*{LastPage}} 
}
\newcommand{\ujtitle}{
    \thispagestyle{title}
%    \noindent\begin{tabularx}{\linewidth}{Xr} % This is a simple tabular environment to align your text nicely 
%    {\large \bf{\@course}} \\
%    \bottomrule % \hline produces horizontal lines.
%    \end{tabularx}    
%    \vspace*{0.3cm}

    \begin{center}
        \let \footnote \thanks
        {\huge\bf\textsc{\@course}}\\[3mm]
        {\@author\@thanks}\\[3mm]
        {\@date}
    \end{center}

    \vspace*{0.3cm}
}
\newcommand{\mymaketitle}{
    \thispagestyle{title}
    \vspace*{-15mm}
    \noindent\begin{tabularx}{\linewidth}{Xr}
                                & \textsl{National Taiwan University} \\
        \large\textbf{\@semester,} & \textsl{Department of \@department}         \\
        \large\textbf{\@course} & \textsl{Prof. \@teacher}\Bstrut\\
        \bottomrule
    \end{tabularx}

    \vspace*{10mm}

    \begin{center}
        {\huge\textsc{\@assignment{}}}\\[6mm]
        {\@author}
    \end{center}

    \vspace*{1cm}
}

% page style for contents
\pagestyle{fancy}
\fancyhf{}
\setlength{\headheight}{13.6pt}
\lhead{\small {\@course}}
\rhead{\small {\@me}}
\cfoot{\small {Page {\thepage} of \pageref*{LastPage}}} 

% Style for part, section, and subsection
% Part
% Bookmarks
\hypersetup{
    bookmarksnumbered=true
}

% Reference nome for task and subtask
\crefname{section}{\sectionname}{\sectionnamepl}

\makeatother

%%%%%%%%%%%%%%%%%%%%%%%%%%%%%%%%%%%%%%%%%%%%%%%%%%%

% \everymath{\displaystyle} % display all math symbol as DISPLAY MODE

% commands for spacing
\newcommand{\Tstrut}{\rule{0pt}{4mm}}         % = `top' strut
\newcommand{\Bstrut}{\rule[-2mm]{0mm}{0mm}}   % = `bottom' strut

% Paired Delimiters {}, (), []
\providecommand\given{} % so it exists
\newcommand\SetSymbol[1][]{
   \nonscript\,#1\vert \allowbreak \nonscript\,\mathopen{}}
\DeclarePairedDelimiterX\Set[1]{\lbrace}{\rbrace}%
 { \renewcommand\given{\SetSymbol[\delimsize]} #1 }
\DeclarePairedDelimiterX{\Bkt}[1]{[}{]}{#1}
\DeclarePairedDelimiterX{\Paren}[1]{(}{)}{#1}
\DeclarePairedDelimiterX{\Abs}[1]{\lvert}{\rvert}{#1}
\newcommand{\PR}[1]{\Pr\Bkt*{#1}}
\newcommand{\overbar}[1]{\mkern 1.5mu\overline{\mkern-1.5mu#1\mkern-1.5mu}\mkern 1.5mu}

%\newcolumntype{D}{>{$(}l<{)$}}
\newcolumntype{R}{>{\displaystyle}r}
\newcolumntype{C}{>{\displaystyle}c}
\newcolumntype{L}{>{\displaystyle}l}

%%%%%%%%%%%%%%%%%%%%%%%%%%%%%%%%%%%%%%%%%%%%%%%%%%%

\usepackage{amsfonts}
\usepackage{amsthm}

%%%%%%%%%%%%%%%%%%%%%%%%%%%%%%%%%%%%%%%%%%%%%%%%%%%

% Declare operator and useful math command
\DeclareMathOperator{\EOp}{\mathbb{E}}
\newcommand{\E}[1]{\ensuremath{\EOp\Bkt*{#1}}}
\newcommand{\R}{\mathbb{R}}
\newcommand{\N}{\mathbb{N}}
\newcommand{\Q}{\mathbb{Q}}
\newcommand{\Z}{\mathbb{Z}}
\newcommand{\PS}[1]{\mathcal{P}(#1)}
\newcommand{\ve}{\varepsilon}
\newcommand{\es}{\emptyset}

\newcommand{\sst}{\subset}
\newcommand{\sse}{\subseteq}
\newcommand{\spt}{\supset}
\newcommand{\spe}{\supseteq}

\newcommand{\ie}{\text{ i.e., }}
\newcommand{\st}{\text{ s.t.  }}
%\newcommand{\and}{\text{and}}

% Declare delimiter
\DeclareMathDelimiter{\lVert}
  {\mathopen}{symbols}{"6B}{largesymbols}{"0D}
\DeclareMathDelimiter{\rVert}
  {\mathclose}{symbols}{"6B}{largesymbols}{"0D}
\DeclarePairedDelimiter\norm{\lVert}{\rVert}
%\DeclarePairedDelimiter\vec{\langle}{\rangle}

%%%%%%%%%%%%%%%%%%%%%%%%%%%%%%%%%%%%%%%%%%%%%%%%%%%

\theoremstyle{plain}
\newtheorem{corollary}{Corollary}
\newtheorem{proposition}{Proposition}
\newtheorem{claim}{Proposition}
\crefname{claim}{Proposition}{Proposition}

%%%%%%%%%%%%%%%%%%%%%%%%%%%%%%%%%%%%%%%%%%%%%%%%%%%

%\setCJKmainfont{思源宋體 TW}
%\setmainfont{思源宋體 TW}

%%%%%%%%%%%%%%%%%%%%%%%%%%%%%%%%%%%%%%%%%%%%%%%%%%%

% define my color
\definecolor{myred}{RGB}{192,46,46}
\definecolor{myblue}{RGB}{7,75,164}
\definecolor{mygreen}{rgb}{0.11,0.7,0.02}

%%%%%%%%%%%%%%%%%%%%%%%%%%%%%%%%%%%%%%%%%%%%%%%%%%%

% Use listing package to display different programming language
% Use xcolor package to display syntax color for programming language
% Usage: \env{lstlisting}[language=LANGUAGENAME]
% Common language name: Python, bash, C, C++, R, sh, make, Matlab

\usepackage{listings}

%%%%%%%%%%%%%%%%%%%%%%%%%%%%%%%%%%%%%%%%%%%%%%%%%%%

\begin{document}
%\begin{CJK*}{UTF8}{bsmi}
\course{Online Learning Behavior, Peer Effect, and Education}
\title{\huge \bf \textsc{Online Learning Behavior, Peer Effect, and Education}}
\semester{Spring 2022}
%\teacher{}
\department{}
\date{\today}
\author{Yu-Chieh Kuo\thanks{Department of Information Management, National Taiwan University.
\textcolor{black}{\href{mailto:ujkuo@ntu.im}{ujkuo@ntu.im}}}}
%\affil[$\dagger$]{Department of Information Management, National Taiwan University}
\maketitle

\begin{abstract}
    %How does the online behavior affect the learning outcome when students learn in the online education
    %platform? Can we detect and specify the communities by their online behavior?
    %This research project aims at answering these questions.

    Sociologists and economists are interested in examining the relationship between
    the characteristics and peer effects of students and their learning outcomes,
    which generate thousands of literature among the past half centry.
    %however, a few attensions are put in the
    %relationship between the learning behavior and the outcomes.
    %In the past, no adequate data is available to analysis such relationship. Nevertheless,
    Additionally, 
    with the widespread adoption of the online learning platforms, 
    more and more behavior data is collected
    by the platform, including the learning behavior such as the clicking, speed-up or replay,
    and posts on the forum.
    Although data collection raises in this big data era, 
    the behavior data doesn't stimulate many research involved,
    and few attentions are received in the
    relationship between the learning behavior and the outcomes.
    What is the impact of different online time distribution and different learning patterns?
    Does the friend share a similar online learning behavior or students with the 
    similar online learning behavior are easier to get familiar with?

    In this research project, 
    we propose a behavior-driven method that provides us to dive into the new era.
    %which provides us a chance to identify the behavior among students.
    The contribution of this research is threefold. 
    First, we present a novel research direction for student's academic outcomes
    by leveraging the student's online behavior data.
    We propose several activities as a possible metric to compare the similarity between
    leaners and cluster learners based on the similarity to confirm 
    whether and how learner behaviors affect their learning outcome.

    Second, we empirically analyze the actual student behavior in the online
    education platform to explore the correlation 
    and the causality for their academic outcome.
    We examine the collected data from NTU COOL,
    an online platform providing professors and students 
    at National Taiwan University to hold courses and learn online.
    Not only to utilize the online website data, but we also plan to collect the 
    student's friendship network to represent the network dynamics and
    study whether the learning behavior is relative
    to the friendship or whether students tend to have friends 
    with similar learning behavior.

    Third, we nominate learning suggestions for both learners and instructors. By
    observing learners' online learning behavior, we can detect possible learning problems
    and help learners better attend the course timely.
    Moreover, this research can be extended to the different age groups to 
    help schools improve their education policies and teaching plans.
\end{abstract}

\newpage

\section*{Long-term Impact}
This research project aims to provide a critical theoretical and empirical 
analysis of online learning behavior, which receives relatively sparse attention
from the researchers.
Regarding the academic contribution, first, we pave the way to 
associate the data era with classical social network research.
Several types of research are conducted to investigate the study outcomes
and the network effects on education.
Nevertheless, due to the difficulty of data collection
in the past, economists tended to emphasizes the characteristics of learners
(\citet{eduPeer}; \citet{idenEduPeer}; \citet{eduPeerBook}) and
no empirical evidence unveiled the mystery behind the learning behaviors.
As the broad adoption of online technologies in recent years, especially
the large-scale epidemics accelerates the acceptance of online education,
millions of learners study different subjects online and leave tons of
online learning behavior marks.
The research's contribution is paramount to both economics and 
educational science, since we leverage the direct behavior data from students
to track and document the impact of different behaviors.
Such a multi-disciplined analysis across the economics, educational science, and computer
science in the framework of network effects is 
currently inexistent, and our methodology may elucidate a new research direction.

Second, we provide a methodological contribution to the future analysis.
It is widely believed that longer studying time and different learning habits
impact academic achievement
(\citet{time}; \citet{time2}; \citet{time3}). Economists and sociologists agree
that people do not live isolated, and peer interactions significantly
influence students (\citet{dynamic}).
We use the stochastic actor-based model and the spatial autoregressive (SAR) model 
for social interaction to study the network dynamics and network effects of
learning behaviors.
We hope the methodology will encourage researchers to explore how different aspects
(e.g., different age groups, cultures, social perception, or the degree of 
importance of study) 
shape diffusion.

In regard to the policy contribution, education is an essential concern of children, parents,
schools, and countries.
As we discover more and more pieces of knowledge, we have no choice but to be forced to
learn more and more. This seems to imply that probing a more effective and efficient
learning method will be indispensable.
By analyzing learning behavior, teachers and students may gain insights
into the role of learning behavior and integrate these insights into their teaching
and studying. In addition, considering the network effects and dynamics
allows us to better understand how socialeconomics networks are involved in our daily life.
The potentials and observations reported here help to clarify socioeconomic network power,
which is worthy of further investigation and extension to other fields.

\newpage

\section*{Objectives}
This research contains the following objectives:
\begin{enumerate}
    \item Propose a theoretical model that captures the similarity of behaviors
        for the online education platforms as a metric to cluster students.
    \item Explore the relationship and effects between the characteristics and the behaviors
        and understand whether students with the similar characteristics result in the similar
        behaviors.
    \item Explore whether the friendship network results in similar behavior and vice versa.
        For example, do the night type students tend to have a friend with the same behavior, or
        do late-submission type students affects the early-submission type.
    \item Construct an empirical approach based on the behavior data and explore the relationship between
        behavior and academic outcomes.
    \item Provide a strategic guide based on the behavior data 
        for learning behavior and a study plan for students and instructors. For example,
        teammate searching and team allocation.
\end{enumerate}

\newpage

\section{Introduction}

Sociologists and economists agree that individual agents interact
in huge networks, and the behavior of which is affected by the network
externalities (\citet{nwExtern}), structure(\citet{struct}), and their peers.
Peer effects are commonly observed in the education (\citet{eduPeer}; \citet{idenEduPeer};
\citet{eduPeerBook} and \citet{school}), management (\citet{entre}), health issues
(\citet{smoke}),
and policy decision (\citet{author2} and \citet{author1})
in both theoretical and empirical studies.
All in all, these works of literature to date seem to confirm the existence and impact
of social network effects.
%Meanwhile, literatures in learning and education have documented the identical attention
%in network effects. 

Moreover, with the rapid development, widespread and
considerable adoption of online technologies,
millions and millions of individuals immensely use online social networking (OSN)
sites such as Facebook (\citet{FB}), Twitter, and Instagram.
OSNs attract numerous users by
allowing them to maintain connections with others and present themselves
in the virtual communities (\citet{FB}).
A great number of usages results in appreciable data collections by platforms for
not only users' personal information (\citet{data} and \citet{digi}) but users' online behavior;
the websites and applications record users' online time, click behaviors, 
usage frequency, and posts and comments in the forum.
Such online behavior extraction is hugely adopted in recommender systems
(\citet{rec}), prediction by machine learning (ML) methods (\citet{CTR} and \citet{click}).
The applications are not limited to the business field,
\citet{mooc} use actions and timestamps taken by the anonymized users on a popular MOOC 
platform\footnote{\url{https://www.mooc.org}} 
and construct a dynamic network to predict when a student will drop out from
a course.

Not only do computer scientists and business companies notice the development
of OSNs and online behavior data, but teachers and educational scholars also
attempt to delve into them.
\citet{selfas} cluster students in three behavioral patterns
and find that students who frequently take online assessments after class tend to
achieve a higher examination score than those who did not.
\citet{moodle} attempt to extract and visualize students' learning behavior
from action logs recorded by Moodle\footnote{\url{https://moodle.org}}, 
a free and open-source learning management system.
These kinds of literature leverage online behavior data to perform the purpose.

Tons of online behavior data is collected nowadays; however,
they seem to receive a little attention from economists, although data collection,
data analysis, and ML methods are common and trendy in the field of Computer Science (CS).
We, economists, have cared about whether and how the characteristics 
result in the economic outcome (\citet{char}) or how personal characteristics
affect individual behavior for several decades.
For example, \citet{risk} investigated three datasets and found that
civil servants are more risk-averse than private-sector employees, 
and women are more risk averse than men.
In this big data era, online platform services are embedded in daily life.
It is worthwhile to link and explore the relationship between the behavior,
the peer effects, and the economic outcome, especially in education.
We want to understand how different behaviors lead to different learning outcomes.
Do students who tend to watch course videos late at night achieve a higher grade?
What is the impact of different online time distribution and different learning patterns?
Does the friend share a similar online learning behavior or students with the 
similar online learning behavior are easier to get familiar with?
Will students who procrastinate
in working and submitting the assignments have a significant lower grade? 
Do students tend to speed up the course videos have a higher grade?
That's what we want to scrutinize.

The first objective of this research is to perform a theoretical structure
to cluster students via their online learning behavior and then conduct an
exhaustive and detailed empirical analysis to examine it.
Classical methods include K-means clustering (\citet{kmeans} and \citet{kmeansnew}), 
hierachical clustering (\citet{hier}), 
and modularity-based methods (\citet{mod} and \citet{mod2}). 
In addition, tens of algorithms are applied in this field, such as the Louvain algorithm
(\citet{luv}).
In this context, how to describe online learning behavior as a metric to determine the similarity
between two users is the first priority.

The second objective of the research
is to illuminate the darkness between peer effects and behaviors.
The numerous literature abounds with theoretical model and empirical results 
on revealing the impact of peer effects 
but rarely devote any time to a systematic study of the influence of
peer effects on online learning behaviors.
Moreover, the influence of dynamic network evolution on online learning behavior
is also essential.
\citet{dynamic} investigate a cohort of 226 engineering students starting their
undergraduate studies and use statistical models for 
dynamic network data to investigate the cohort's social network formation processes.
They find that friendship ties crucially affect academic success and suggest the
university promotes the development of positive relationships.
Students strike up a friendship through time, and 
we want to examine how network dynamics and friendship ties affect the learning behavior,
or conversely, how the learning behavior affects network dynamics and friendship.

The third objective of the research is to provide learning suggestions to students and instructors.
With a thorough study of online learning behavior and the corresponding insights,
we may offer a better study plan to both students and instructors. For example,
assume that the procrastination may lead to a terrible grade, the instructor can find students
in the procrastination and help them earlier.
Online learning behavior may also reveal students' preferences. Students who tend to 
finish assignments early may prefer joining a team with a similar type
of teammates. We want to figure out the secret behind the behavior.

With these questions in mind, the research results may help raise the economists' and 
educational scolars' attention for exploring the possibility of learning behavior,
and also help students and teachers better acknowledge the role of learning behavior in studying.

%leverage and combine the online behavior data into social network analysis to
%examine the network effects on the behavior.

%How human beings behave receives attension from discinplilies varing from economics,
%and pshcology to siocialogy and learning science. Thonsands of researchers try to catch the
%mistery from human behavior.
%In economics field, behavioral economics studies the effects on the decision of 
%individuals and institutions, and how these decisions vary from which derived by
%classical economic theorys.
%\citet{BE} provides a general and distinguished introduction to this field.
%Economists, phycologists, and sociologists agree that people do not live isolatedly.
%People live in a social network, and we interact and are affected by differents. 
%We call such effects the peer effect (literature).
%Literatures have documented how peer effects and network externalities hugely
%change the academic outcomes (LR), production (LR), blablabla.
%
%After the Internet was generated, people's life cannot leave without it. Tons of literatures.
%
%With the wide adoption of online platform, bla
%
%Under such perspective,
%
%This paper axiomatically analyzes a decision problem of institutional choice in
%
%\[
%    \mathscr{N,A,O}=N,A,O\Set{1,\cdots,n}
%\]

\section{Research Methodology}
In this section, we organize the current methods to 
capture the behavior statistics, derive the similarity, detect communities,
represent the network dynamics, and evaluate the influence of behavior.


\subsection{Behavior Extraction}
Online platforms record several different actions from users, and it is 
crucial to extract behaviors from data. 
For example, NTU COOL\footnote{\url{https://cool.ntu.edu.tw}} documents
total activity time,
activity (page views) by date, the number of communications, 
submission time, and assignment grade. If the instructor has uploaded course
videos to NTU COOL, the videos' completion rate, activity
(fast forward, rewind, and pause clicks), playing speed, and the number of video comments
is also recorded.

Not only the web activities but learners’ in-time motions are also also
worthy of track and examination. 
In the experimental economics,
eye-tracking technology has been used to design, implement, and analyze an experiment 
using this technology to study economic theory (\citet{eye}).

These activities uncover the students' online learning behaviors and learning preferences.
Further research is needed to colloborate and complete the findings in this area.

\subsection{Similarity}
\citet{behavclus} use the time-stamped series of completion times for tasks 
and the time-stamped data of page-clicks
from each leaner in the dataset as a metric
and adopt a dynamic time wrapping (DTW) kernel (\citet{dtw} and \citet{svmdtw}) 
to calculate a pairwise similarity between
time-series of learner actions, construct the leaner similarity graph, 
and then apply the community detection to cluster
learners according to their similar time-series behaviors.

Given two leaners $i$ and $j$ in the set of $N$ leaners $\mathcal{D}=\Set{1,\cdots,N}$
and the time-series of actions $S_i, S_j$, where $S_i = [s_{i1}\ s_{i2} \cdots s_{in}]^T$
and $S_j=[s_{j1}\ s_{j2}\cdots s_{jm}]^T$; that is, the compared time-series can be in different length.
Common approaches for sequence analysis exploit $L_{p}$-norms between $S_i$ and $S_j$
for its convenience and fast computation; however, such an approach is a one-to-one
mapping, which often neglects and misaligns sequential patterns.
Dynamic time wrapping can overcome this problem and
have advantages over $L_p$-norms in its elastic and robust matching.
The DTW cost of two time-series $S_i,S_j$ at $(p,q)$ is
\[
    c(p,q)=\norm{s_{ip}-s_{jq}}^2+\min\Set{c(p-1,q),c(p-1,q-1)
    ,c(p,q-1)},
\]
and the DTW similarity is the longest path between $S_i,S_j$,
i.e., $\delta(S_i,S_j)=c(n,m)$.

Moreover, the DTW similarity kernel is defined as
\[
    k\left(S_i, S_j\right) = \exp\left\{\frac{-\delta(S_i,S_j)}{\sigma^2}\right\}.
\]
We use the DTW similarity kernel to represent the similarity (or distance) between
learner $i,j$. Therefore, the similarity matrix $Y$ is a $N\times N$ matrix where the value of element
$y_{ij}$ is $k(S_i, S_j)$, and $Y$ can be treated as a weighted and fully connected adjacency matrix.

The calculation of DTW similarity may become computational consuming with the
growing number of leaner or longer time series.
Therefore, dimensionality reduction methods
are imperative and can be implemented to accelerate the computation. 
Several methods are proposed and used in ML applicationd, such as principal component analysis (PCA),
singular value decomposition (SVD), or relaxed minimum spanning tree (RMST).
As we do not need to connect all nodes (learners) in the graph, we can use
the RMST algorithm to prune the similarity matrix (see \citet{rmst}).

Given the similarity matrix $Y$, we define the {\it dissimilarity} matrix $Z$ with
$z_{ij}=1-y_{ij}$. Next, we find the maximal weight in $Z$ along the maximum spanning tree
path as
\[
    b_{ij} = \max\left\{z_{ik},z_{kh},\cdots,z_{cj}\right\}.
\]
If $b_{ij}$ is much smaller than $z_{ij}$, we discard the direct link between $i,j$;
if $z_{ij}$ and $b_{ij}$ are comparable, we leave the link between $i,j$ if
\[
    b_{ij}+\gamma(d_i+d_j) > z_{ij},
\]
where $d_i=\min_k z_{ik}$ and $\gamma$ is a parameter. Such RMST method merges
local and global impact of the data to sparsify the network.
Meanwhile, pruning the network help community detection methods speed up and 
save computation time.

\subsection{Clustering}
Community detection methods allow us to detect groups with similar properties 
and extract groups for various reasons and interests. Tens of methods can be used to 
separate the nodes of a graph into subgraphs, including K-means clustering, 
hierarchical clustering, modularity-based methods, and Markov stability and vector partitioning
(\citet{ms}). This research can use either K-means clustering and Markov stability or vector partitioning
methods. The former can specify the number of target clusters,
and the latter is a generalized method that
uses the diffusion of a Markov process on the graph to unveil the subgraph
at all scales. We can shift between two methods depending on the purpose. More specifically,
if we merely need two subcommunities to separate learners, we can adopt K-means clustering.

We aim at partitioning students in different group based on
their extracted behavior properly and revealing the interactions in the group members.
A general iterative cluster approach is to minimize
the distance between each node and the center of group, and update the center iteratively.
Given a $d$-dimensional set of $N$ students properties $\mathscr{X}=\{x_1,\cdots,x_N\}$
and $K$ groups, we define a $d$-dimensional set of $K$ clusters $\mathscr{C}=\{c_1,\cdots,c_K\}$,
the objective function of K-means clustering algorithm optimizes
\[
    \begin{array}{C}
        f^{\text{KM}}(\mathscr{X,C})=\sum_{i=1}^N\min_{j\in\{1,\cdots,K\}}\norm{x_j-c_j}^2.
    \end{array}
\]
A membership $0\leq m^{\text{KM}}_j(c_j|x_i)\leq1$ defines the proportional of data $x_i$ belonging to
the group $j$ with the center $c_j$, and the weight $w(x_i)>0$ reflects the influence in 
updating the new center in the group.
In the K-means scenario, it is
\[
    \begin{array}{RCL}
        m_j^\text{KM}(c_{j^*}|x_i) & = & 
        \begin{cases}
            1 & \text{if } j^*=\argmin_j\norm{x_i-c_j}^2 \\
            0 & \text{otherwise,}
        \end{cases} \\
        w^\text{KM}(x_i) & = & 1.
    \end{array}
\]
Another common Bayesian approach is the Gaussian expectation-maximization (GEM).
It minimizes the objective function of GEM
\[
    \begin{array}{C}
        f^{\text{GEM}}(\mathscr{X,C})
        =-\sum_{i=1}^N\log\left(\sum_{j=1}^kp(x_i|c_j)p(c_j)\right),
    \end{array}
\]
where $p(x_1|c_j)$ is the probability of $x_i$ conditional on that it's
generated by the Gaussian distribution with center $c_j$,
and $p(c_j)$ is the prior of $c_j$. The membership and the weight in the GEM
scenario is
\[
    \begin{array}{RCL}
        m_j^{\text{GEM}}(c_j|x_i) & = & 
        \frac{p(x_i|c_j)p(c_j)}{p(x_i)} \\
        w^{\text{GEM}} & = & 1.
    \end{array}
\]
Lastly, the steps of iterative algorithm is:
\begin{enumerate}
    \item Initialize the center $\mathscr{C}$.
    \item For each $x_i$, compute its membetship $m_j(c_j|x_i)$
        and weight $w(x_i)$.
    \item Recompute each center $c_j$ after assigning $x_i$ to
        one group.
    \item Repeat the step 2 and 3 until convergence.
\end{enumerate}
The time complexity of clustering algorithms above is $\mathscr{O}(nkd)$.

\subsection{Dynamic Evolution}
%cite:https://reader.elsevier.com/reader/sd/pii/S0378873309000069?token=77E80FB6F5EEAFF76EAB13EC5D4CF697B4FA625800327C6FB979B2A1EAC1E951664576D5A0C144E2EA2D582EA14E9F42&originRegion=us-east-1&originCreation=20220619041405
Network dynamics can be studied by the spatial dynamic panel data (SDPD) model
(\citet{sdpd})
and the stochastic actor-based model (\citet{sto} and \citet{storev}).
The stochastic actor-based model has been studied in
several fields, varying from statistics to psychology and medicine. 
It can represent numerous
influences on network change, allow to estimate
parameters expressing such influences, 
and test corresponding hypotheses.
The difficulty of network dynamics results from multi changes in network structure
and the individual behavior. Individual behavior is often not only 
influenced by networks but also imposes influence on networks.
Researchers impose several assumptions on the problem and adopt a continuous-time Markov process
to ease the problem. It is commonly assumed that the changing system consisting of 
network and behavior follows a Markov process, and no more than one network variable
or behavior variable can change at any moment $t$.

Given a changing network on $n$ individuals $\mathscr{G}$
\footnote{A network $\mathscr{G}$ is a $n\times n$ adjacency matrix.}
and the vector of behavior state of
individual $i$ $Z_i=[z_i(1), z_i(2),\cdots,z_i(t)]$, 
$\lambda_i^\mathscr{G},\lambda_i^Z,f_i^\mathscr{G}, f_i^Z$
are rate functions for a Possion process 
and objective functions of $\mathscr{G}$ and $Z$.
We first present the objective function of
individual $i$ for the network
and the behavior:
\[
    \begin{array}{CCC}
        f_i^\mathscr{G}(g,z)=\sum_{j}\beta_j^\mathscr{G}s_{ij}^\mathscr{G}(g,z)
        %\equiv s_k^\mathscr{G}(g,z)\sum_k\beta_k^\mathscr{G}
        & \text{and} &
        f_i^Z(g,z)=\sum_{j}\beta_j^Zs_{ij}^Z(g,z),%\equiv s_k^Z(g,z)\sum_k\beta_k^Z.
    \end{array}
\]
where $s_{ij}^\mathscr{G}(g,z)$ and $s_{ij}^Z(g,z)$ are utility functions
depending on the behavior of the focal individual $i$ and the network,
which also provides the micro-foundation.

Since the chances for change are for either the network or the behavior of the individual,
given the individual $i$ has an opportunity for change in behavior, the current
state value $z_i(t)=s^0$, the option for the next possible state $z^0-1,z^0,z^0+1$,
the probability of state transition is 
\[
    p_i^Z(\beta,g,z^0,z)= \begin{cases}
        \frac{\exp\left\{f_i^Z(\beta,g,z^0,z)\right\}}{\sum_{\delta=-1}^1\exp\left\{f_i^Z(\beta,g,z^0,z^0+\delta)\right\}}
        & \text{if } z=z^0+\delta, \, \delta\in\{-1,0,1\} \\
        0 & \text{otherwise.}
    \end{cases}
\]
We can use the average similarity effect 
\[
    s_{i}^Z(g,z)=g_{i+}^{-1}\bigg\rvert_{t-1}\sum_{j}g_{ij}\bigg\rvert_{t-1}
    \left(\text{sim}_{ij}^Z-\overline{\text{sim}}^Z\right)\bigg\rvert_{t}
\]
to capture the effect of network incluence, where
\[
    \text{sim}_{ij}^Z=\frac{1-|z_i-z_j|}{\max_{ij}|z_i-z_j|}
\]
and $\overline{\text{sim}}^Z$ is the mean of similarity. In addition, the homophily
from certain behavior can be captured by the similarity effect
\[
    s_i^\mathscr{G}(g,z)=\sum_{j}g_{ij}\bigg\rvert_{t}
    \left(\text{sim}_{ij}^Z-\overline{\text{sim}}^Z\right)\bigg\rvert_{t-1}.
\]

Furthermore, we can use the Generalized Method of Moments (\citet{dyes})
and Bayesian Markov chan Monte Carlo approach (\citet{dybaye})
to estimate parameters for network and behavior $\beta^\mathscr{G}, \beta^Z$.

\subsection{Social Interactions Model}
Now that we can cluster learners from their behavior similarity, and it is natural to examine
what behavior is more effective and efficienct for learners.
The traditional model for studying peer effects is the
linear-in-linear model (maskin 1993). 
However, the linear-in-linear model suffers from the reflection
problem, preventing researchers from identifying endogenous and contextual effects.
Fortunately, the spatial autoregressive (SAR) model can overcome the reflection problem.
The network interaction model with the endogenous and
contextual peer effects and the group effects
is specified as
\[
    y_{ig}=\lambda\sum_{j=1}^{m_g}\overline{w}_{ijg}y_{jg}+\beta_1x_{ig}
    +\beta_2\sum_{j=1}^{m_g}\overline{w}_{ijg}x_{jg}
    +\alpha_g+\varepsilon_{ig},\quad g\in\Set{1,\cdots,n_\mathscr{G}},
\]
where $y_{ig}$ is the outcome of interest for the
individual $i$ within the group $g$,
$x_{ig}$ is the independent variable 
for the individual $i$ within the group $g$,
and 
$\overline{w}_{ijg}=\frac{w_{ijg}}{\sum_jw_{ijg}}$,
$w_{ijg}=1$ if individual $j$ is $i$'s friend.
$\lambda,\beta_1,\beta_2$ and $\alpha$ reflect the endogeneity, 
the own effect, contextual (peer) effect, and the fixed group effect.

To probe more deeply the influence of online learning behavior,
we emphasize on the estimation of group effect $\alpha$ to evaluate different
behaviors.



\section{Data Collection}
We collaborate with the
Center for Teaching and Learning Development Digital
Learning Center at National Taiwan University(NTU).
They offer an online teaching and learning platform,
NTU COOL\footnote{\url{https://cool.ntu.edu.tw}}, 
to serve faculty members and students at the university
to use digital technologies and media materials in the course.
NTU COOL team collect several user activities to do the research, including total activity time,
activity (page views) by date, the number of communications, 
submission time, and assignment grades. If the instructor has uploaded course
videos to NTU COOL, the videos' completion rate, activity
(fast forward, rewind, and pause clicks), playing speed, and the number of comments
on the video.

Furthermore,
as the research is relative to personal information,
to protect the rights and welfare of human research
subjects recruited to participate in research activities,
our research will be verified by the institutional review board (IRB) from
the Center for Taiwan Academic Research Ethicals.

In addition to the online learning behavior,
we are also interested in the network dynamics, behavior diffusions, and 
peer effects. Following the \cite{dynamic},
we plan to track a cohort of freshmen and 
document various individual and
socioeconomic background variable to 
conduct the empirical setting to examine that
with the growing links between learners and
the emergence of social networks, whether learners'
online learning behaviors are affected by peers,
and whether learners' behaviors influence the network
emergence and formation.

\bibliography{ref}

%\end{CJK*}
\end{document}
