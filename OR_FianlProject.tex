\documentclass[12pt,a4paper]{article}

\usepackage{amsfonts}
\usepackage{amssymb}
\usepackage{amsmath}
\usepackage{amsthm}
\usepackage{epsfig}
\usepackage{graphicx}
\usepackage{natbib}		% citet, citep
\usepackage{textcomp}
\usepackage{booktabs}
\usepackage{multirow}
\usepackage{fullpage}
\usepackage{authblk}
\usepackage{url}
\usepackage{color}
\usepackage{listings}
\usepackage{pgfplots}
\usepackage{comment}
\usepackage[utf8]{inputenc}

\setlength{\arrayrulewidth}{1mm}
\setlength{\tabcolsep}{18pt}
\renewcommand{\arraystretch}{1.5}

\renewcommand{\baselinestretch}{1.4}

\parskip=5pt
\parindent=20pt
\footnotesep=5mm

\newtheorem{lem}{Lemma}
\newtheorem{prop}{Proposition}
\newtheorem{defn}{Definition}
\newtheorem{cor}{Corollary}
\newtheorem{ass}{Assumption}
\newtheorem{obs}{Observation}
\newenvironment{pf}{\begin{proof}\vspace{-10pt}}{\end{proof}}
% \newtheorem{ques}{Question}
% \newtheorem{rmk}{Remark}
% \newtheorem{note}{Note}
% \newtheorem{eg}{Example}

\newenvironment{enumerateTight}{\begin{enumerate}\vspace{-8pt}}{\end{enumerate}\vspace{-8pt}}
\newenvironment{itemizeTight}{\begin{itemize}\vspace{-8pt}}{\end{itemize}\vspace{-8pt}}
\leftmargini=25pt   % default: 25pt
\leftmarginii=12pt  % default: 22pt

\DeclareMathOperator*{\argmax}{argmax}
\DeclareMathOperator*{\argmin}{argmin}






\title{Operations Research, Spring 2020 (108-2) \\
Modeling the Optimal Vehicle Scheduling for YouBike2.0 for NTU Campus Based on Integer Programming}


\author{Yu-Chieh Kuo B07611039, Yu-Hsuan Chang B05607004, Oliver Dawson F10815022, Benjamin Pfundstein T08701102, Chung-Chia Kuo B05703100, Hung-Wen Tsai B05705007, Cho-Ting Ying, B04705030}
\date{}



\begin{document}


%\begin{titlepage}
\maketitle
%\thispagestyle{empty}

YouBike2.0, a bicycle-sharing system in Taiwan, launched its service on NTU campus in January 2020, where lots of students get around by their own bikes everyday. Having rental sites spread across the campus, YouBike2.0 provides an alternative option for students and tourists alike to get around conveniently. To encourage the use of the bike-sharing system, it is important to make sure that there are enough bikes for rent as well as vacancy for returning of bikes at each site.

Our goal is to find the best routes for YouBike2.0 dispatchers to travel between the rental sites on their trucks, within a certain time limit, considering the need for bike reallocation at each site.

We came up with two tasks. One is to maximize the number of bikes reallocated, and the other is to minimize the sum of the difference between actual and ideal number of bikes at each site. We formulate two integer programs and build models with AMPL. We obtain our data from Wei-Cheng Huang, the student representative from department of engineering, who has been devoted to the operation of YouBike2.0 on NTU campus, and from the YouBike official website. The result shows the optimal route to travel at different times of a day, where there are varying demands at each site. At the end, we analyze our results and give them reasonable explanations.


\section{The Preliminary Task}
On NTU campus, there are 54 YouBike2.0 rental sites with 984 racks and about 500 to 600 YouBike2.0s available every day. Pickup trucks accommodate YouBike2.0s at different times to ensure the availability at all spots. To find the distance between different spots, we made a proxy that the distance between each spot can be found by calculating the norm from their latitude and longitude directly. 

We denote the starting and the finishing spot as $O$, the set of spots $S$, the total number of spots $N$, the number of racks $q_i$ at spot $i$, the distance $c_{ij}$ between spot $i$ and $j$, the number of available YouBike2.0s $a_i$ at spot $i$, the number of trucks $K$, the capacity of a truck $Q$, the average velocity of a truck $v$, the operation time of moving a YouBike2.0 $m$, the time limit $L$. Our target is to maximize the number of reallocated YouBike2.0s within the given time limit. We calculate the difference between the number of available YouBike2.0s $a_i$ and the number of racks $q_i$ at spot $i$ with a ratio $r$, which can be written as $w_i = |q_ir - a_i|$, to represent the demand at spot $i$. We assume $r=0.5$ is the ideal ratio.

Let spots be a complete directed graph $G(V,E)$, $x_{ij}^k$ be 1 if truck $k$ travels directly from spot $i$ to spot $j$ or otherwise 0, $y_i^k$ be a binary variable equal to 1 if spot $i \in S$ is visited by the truck $k$'s route or 0 otherwise, $u_i$ be the order of passing spots. Our objective function is
\[
\max \ \sum_{i \in S}\sum_{k=1}^K w_iy_i^k
\]
To ensure that one arc enters and one arc leaves each visited spot, we have
\[\begin{split}
    \quad & \sum_{i \in S} \sum_{k=1}^K x_{ij}^k = y_j^k \ \forall j \in S \\
    \quad & \sum_{i \in S} \sum_{k=1}^K x_{ji}^k = y_j^k \ \forall j \in S \\
    \quad & \sum_{k=1}^Kx_{ii}^k = 0 \ \forall i \in S
\end{split}\]
Thus, to ensure there is no repeated route, we have
\[\begin{split}
    \quad & \sum_{k=1}^K y_1^k \leq K \\
    \quad & \sum_{k=1}^K y_i^k \leq 1 \ \forall i \in S/\{1\}
\end{split}\]
In order to eliminate subtours, we use the method taught in class, the part about vehicle routing problems. Let $u_i$ represent the order of passing nodes. More precisely, $u_i=n$ if node $i$ is the $j$th node to be passed in a tour. There are $N$ spots on NTU campus, therefore, we can add the following constraints
\[
u_1^k = 1 \ k = 1,...,K
\]
\[
2 \leq u_i^k \leq N \ \forall i \in S/\{1\}, \  k = 1,...,K
\]
\[
u_i^k-u_j^k+1 \leq (N-1)(1-x_{ij}^k) \ \forall(i,j) \in S,i \neq 1,j \neq 1, \  k = 1,...,K
\]
Also, the whole accommodation work should be finished in the given time limit.
\[
\sum_{i,j \in S} \frac{c_{ij}x_{ij}^k}{v} + d_{ij} m \leq L
\]
Finally, the integrality constraints defining the nature of the decision variables are
\[
    x_{ij}^k, y_{i}^k \in \{0,1\} \  \forall (i,j) \in S, k=1,...,K
\]
\[
    u_i^k \geq 0 \ \forall i \in S, \ k=1,...,K
\]
Via this integer programming, we can find the optimal path at NTU campus in the given time limit.




\section{Another Task}
Since the former formulation ignores the capacity limit of a pickup truck, we improve the formulation with adding the capacity constraint. In addition, the former formulation only shows the demand of spots, but it doesn't show whether the demand is to deliver or to pickup. Also, to maintain the best ratio between the amount of YouBike2.0 and racks. In this case, we assume the ratio $r=0.5$. Therefore, we modify it in this program.

We modify the original notation by replacing the $w_i$, with the demand for delivery being $d_i$, the demand for pickup $p_i$ for spot $i$, and adding a binary parameter $z_i$ to determine whether the delivered demand or the pickup demand for spot $i$, $z_i$ is equal to 1 if there is the delivered demand and 0 otherwise. But in this version, we can only deal with one-truck problems due to the difficulty of expanding it to a multi-truck problem. Decision variables need to be modified as well. Let $x_{ij}^n$ be 1 if the truck travels directly from spot $i$ to spot $j$ with $n$ YouBike2.0s, $y_i$ be a binary variable equal to 1 if spot $i \in S$ is visited by the truck or 0 otherwise, $u_i$ be the order of passing spots, $t$ be used to linearize the absolutely value. Our objective function is to minimize the difference between the amount of YouBike2.0 that are in the racks and half of racks that exist for spot $i$ , that is
\[\begin{split}
\min \quad & \sum_{j = 2}^N[z_j(d_j - (\sum_{i=1}^N\sum_{n = 0}^Qnx_{ij}^n - \sum_{k=1}^N\sum_{m=0}^Qmx_{jk}^m)) \\
\quad & + (1-z_j)(p_j - (\sum_{k \in S}\sum_{n=0}^Qnx_{jk}^n -\sum_{i =1}^N\sum_{m=0}^Qmx_{jk}^m))] \\
\quad & +(1-z_1)(p_1 - \sum_{k=2}^N\sum_{n=0}^Qnx_{1k}^n) + z_1t
\end{split}
\]
To ensure that the amount of YouBike2.0 in our trucks after leaving a spot is greater or equal to the amount of YouBike2.0 that were contained in a truck before arriving to the spot (minus the demand on the spot, yet, not more than the amount of bikes on the truck upon arrival, plus, the surplus of bikes on the spot), we have to define the following demand constraints (so we always have enough bikes to distribute to the next spot).
\[\begin{split}
    \sum_{i=1}^N\sum_{n=0}^Qnx_{ij}^n - d_j \leq \sum_{k=1}^N\sum_{m=0}^Qmx_{jk}^m \ j=2,...,N \\
    \sum_{i=1}^N\sum_{n=0}^Qnx_{ij}^n + p_j \geq \sum_{k=1}^N\sum_{m=0}^Qmx_{jk}^m \ j=2,...,N
\end{split}
\]
To check the amount of YouBike2.0 the truck can move from the starting spot and to ensure the demand for the starting spot is non-negative, let $h = \min\{p_1, Q\}$, then we have
\[
\sum_{j=2}^Nx_{ij}^h = 1
\]
\[\begin{split}
    \quad & t \geq \sum_{i=2}^N\sum_{m=0}^Qmx_{i1}^m - d_1 \\
    \quad & t \geq d_1- \sum_{i=2}^N\sum_{m=0}^Qmx_{i1}^m
\end{split}
\]
Also, the whole accommodation work should be finished in the given time limit, and we need to consider the operation time of moving YouBike2.0s. Hence, we have
\[\begin{split}
\quad & \sum_{j=2}^N[z_j(\sum_{i=1}^N\sum_{n=0}^Qnx_{ij}^n-\sum_{k=1}^N\sum_{m=0}^Qmx_{jk}^m)+(1-z_j)(\sum_{k=1}^N\sum_{n=0}^Qnx_{jk}^n-\sum_{i=1}^N\sum_{m=0}^Qmx_{ij}^m)] \\
\quad & + \sum_{i=1}^N\sum_{j=1}^N\sum_{n=0}^Q\frac{c_{ij}x_{ij}^n}{v} + (1-z_1)\sum_{k=2}^N\sum_{n=0}^Qnx_{1k}^n + z_1\sum_{i=2}^N\sum_{m=0}^Qmx_{i1}^m \leq L
\end{split}\]
To ensure that one arc enters and one arc leaves each visited spot, we have
\[\begin{split}
    \quad & \sum_{i=1}^N\sum_{n=0}^Qx_{ij}^n = y_j \ j=1,...,N \\
    \quad & \sum_{i=1}^N\sum_{n=0}^Qx_{ji}^n = y_j \ j=1,...,N \\
    \quad & x_{ii}^n = 0 \ i=1,...,N \ n = 1,...,Q
\end{split}\]
In order to eliminate subtours, we use the method in the slide of IPapp, the part of vehicle routing problems. Let $u_i$ represent the order of passing nodes. More precisely, $u_i=n$ if node $i$ is the $j$th node to be passed in a tour. There are $N$ spots at NTU campus, therefore, we can add the following constraints
\[
u_1 = 1
\]
\[
2 \leq u_i \leq N \ \forall i \in S/\{1\}
\]
\[
u_i-u_j+1 \leq (N-1)(1-x_{ij}^n) \ \forall(i,j) \in S,i \neq 1,j \neq 1, \ n = 0,...,Q
\]
Finally, the integrality constraints defining the nature of the decision variables are
\[
    x_{ij}^n, y_{i} \in \{0,1\} \  \forall (i,j) \in S, n=0,...,Q
\]
\[
t \geq 0, \ u_i \geq 0 \ \forall i \in S
\]

\section{Data Acquisition}

We scrap the data for YouBike2.0 from the official website. With the web crawler, we can acquire the amount of available YouBike2.0 $a_i$ for spot $i$ directly. Also, the data from the official website shows the distance between each spot. Therefore, we own all necessary data sets for this project. 

We select the amount of available YouBike2.0 $a_i$ for spot $i$ in a different time to find a different demand and the truck route. The selecting criteria is based on the timing particularity (If it is a rush hour or not). For example, we select the data in the starting time for first class in the morning and in the afternoon, and the ending time for the last class in the afternoon. We finally select 7am to 9am, 12pm to 1pm, 5pm to 7pm as out data file. Besides, due to the huge loading time to execute the AMPL program in our computers, we eliminate spots with a rack count of 10 or less YouBike2.0 on the NTU campus, which decreases the number of spots from 54 to 32. The distance between selected spots is shown as below.

\begin{figure}[htp]
    \centering
    \includegraphics[width=10cm]{DistanceInSpots.png}
    \caption{The distance in spots}
    \label{fig:galaxy}
\end{figure}




\section{Simulation with the Reality Situation}
\subsection{Task1: Maximize the amount of accommodation}
Now we formulate an integer program and acquire the reality data, and we are able to simulate the truck route with computer program. At first, we deal with the preliminary task of the time limit $L = 60$ mins, the operation time of moving a YouBike2.0 $m = 1$ mins, the average velocity of truck $v=200$ $m/s$, the amount of truck $k=1$. The following figure shows the different routes for different times. In this case, the problem can be perfectly done by AMPL, and we can conclude that the result is reasonable.
\begin{figure}[htp]
    \centering
    \includegraphics[width=8cm]{A1.png}
    \caption{The visualized result}
    \label{fig:galaxy}
\end{figure}

We are going to deal with the advanced case. In our observation, the YouBike2.0 staff can move a YouBike to a truck (and vice versa) in about 20  seconds. To simplify the model, we assume that the operation time of moving a YouBike2.0 is $m=0.5$ minutes. In this case, although we use the computer with i9 CPU and GTX-2080 GPU to compute the model, it still costs too much time to finish it. Hence, we find the approximate solution by AMPL. We use the same computer to simulate for 20 minutes, and we only test the data for 7am, 12pm and 5pm, to find the approximate optimal route. It's reasonable to merely use the data for three different time since these three data sets can simulate the situation of how YouBike2.0 company deals with the accommodation problem before the rush time.
\begin{figure}[htp]
    \centering
    \includegraphics[width=8cm]{A2.png}
    \caption{The visualized result with $k=1$, $m=0.5$}
    \label{fig:galaxy}
\end{figure}

Also, our formulation can simulate the accommodation with multi-vehicles. Due to the restriction of computing resources, we only acquire the results for the amount of truck $k=1,2$ with $m=0.5$ and $k=1,...,3$ with $m=1$ for 7am, 12pm and 5pm. The figure below shows the 2 trucks with $m=0.5$, which is one of our total results.
\begin{figure}[htp]
    \centering
    \includegraphics[width=8cm]{A3.png}
    \caption{The visualized result with $k=2$, $m=0.5$}
    \label{fig:galaxy}
\end{figure}

The following table is to compare the differences between the total amount of accommodation with different parameters.

\begin{center}
\begin{tabular}{ |p{1cm}|p{1.5cm}|p{1.5cm}|p{1.5cm}|p{1.5cm}|    }
\hline
\multicolumn{5}{|c|}{Comparison List} \\
\hline
Time& $k=1$ and $m=1$ & $k=1$ and $m=0.5$ & $k=2$ and $m=1$ & $k=2$ and $m=0.5$ \\
\hline
7am & 53 & 99 & 103 & 187 \\
8am & 54 & None & None & None \\
9am & 53 & None & None & None \\
12pm & 53 & 88 & 103 & 189 \\
1pm & 54 & None & None & None \\
5pm & 53 & 98 & 102 & 189 \\
6pm & 54 & None & None & None \\
7pm & 54 & None & None & None \\
\hline
\end{tabular}
\end{center}

\subsection{Task2: Maintain the best ratio between the amount of YouBike2.0 racked and total racks} % ask how to define the title
The second task is to maintain the ratio between the amount of YouBike2.0 in the racks and the total racks for each spot $i.e.$ to minimize the difference between the amount of YouBike2.0 and racks with the ratio $r=0.5$. In comparison with the difference after we apply our formulation, we can find the effectiveness of our formulation. The given parameters are time limit $L=60$ mins, the operation time of moving a YouBike2.0 $m=0.5$ mins, the average velocity of truck $v=200$ $m/s$, the amount of truck $k=1$, and the capacity of truck $Q=14$.

The figure below shows the route and the amount of accommodation for each spot on the route in different times and the table shows the change of difference after accommodation. To keep the clear figure, we only draw the three routes for 7am, 12pm and 5pm, respectively. Note that the route is a directed weighted graph, the positive weight means from spot $i$ to $j$, how many YouBike2.0 the truck delivers from spot $i$ to $j$ and the negative weight means how many YouBike2.0 the truck picks up from spot $i$ to $j$.
\begin{figure}[htp]
    \centering
    \includegraphics[width=8cm]{A4.png}
    \caption{The visualized result to minimize the difference}
    \label{fig:galaxy}
\end{figure}

\begin{center}
\begin{tabular}{ |p{1cm}|p{1.5cm}|p{1.5cm}|    }
\hline
\multicolumn{3}{|c|}{Comparison List} \\
\hline
Time& Before & After \\
\hline
7am & 230 & 182 \\
8am & 221 & 194 \\
9am & 235 & 200 \\
12pm & 239 & 207 \\
1pm & 257 & 205 \\
5pm & 272 & 184 \\
6pm & 256 & 208 \\
7pm & 237 & 187 \\
\hline
\end{tabular}
\end{center}

\section{Results Analysis}
Instead of merely acquiring results from the computer programming, we want to analyze the result further and try to give the result a reasonable explanation, and find the insufficiency of our model.
\subsection{The relationship between multi-vehicle and the amount of accommodation}
In our results, the amount of accommodation is proportional to the number of trucks, However, The relationship between multi-vehicle and the amount of accommodation may be changed because of scaling problems when the amount of trucks increases. Without the simulation with enough trucks, we cannot determine whether the proportion will not change when the truck count is too high. The proportion may be saturated when there are many operating trucks, and it's difficult for us to execute adequate simulations to find the relationship between the number of truck and the amount of accommodation.
\subsection{Single starting position}
In our model, all trucks start from the same spot. But in reality, it is possible for trucks to start from different spots. Since it is too difficult for us to deal with the multi-starting-position problem, which is more like the real situation, we have no choice but to add the single starting position constraints. In the multi-starting-position problem, trucks can go through more spots to accommodate the YouBike2.0, which may result in a more effective route.


\section{Conclusion}

With the rise of the sharing economy, people gradually change their way of commuting. Although it aroused controversy when YouBike2.0 began to operate at NTU campus, there is no denying that the campus life becomes more convenient due to YouBike2.0. As an NTU student, the most important thing for us is to have a good environment for studying, for attending the college events, and contributing to the development of society. We believe that YouBike2.0 will be more accessible and convenient for us, and that's the main reason we select this topic and do our best to solve it. In this project, we successfully constructed the integer program, and we use the computer program to help deal with solving it, and finally find the reasonable optimal solutions. Although it is not close enough to the reality situation, we believe that this project still provides valuable insights for contemporary issues related to the accessibility of YouBike2.0 commuting.  

\end{document}
