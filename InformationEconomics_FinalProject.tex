\documentclass[a4paper]{article}
\usepackage[top = 2.5cm, bottom = 2.5cm, left = 2.5cm, right = 2.5cm]{geometry} 
\usepackage{amsmath}
\usepackage{amssymb}
\usepackage{amsthm}
\usepackage{mathtools}
\usepackage{tabularx}
\usepackage{booktabs}
\usepackage{authblk}
\usepackage[numbered]{bookmark}
\usepackage[capitalize]{cleveref}
\usepackage{array}
\usepackage{cite}

%%%%%%%%%%%%%%%%%%%%%%%%%%%%%%%%%%%%%%%%%%%%%%%%%%%

\theoremstyle{definition}
\theoremstyle{plain}
\newtheorem{proposition}{Proposition}
\newtheorem{corollary}{Corollary}

\theoremstyle{remark}
\newtheorem*{remark}{Remark}

%%%%%%%%%%%%%%%%%%%%%%%%%%%%%%%%%%%%%%%%%%%%%%%%%%%

% Paired Delimiters {}, (), []
\providecommand\given{} % so it exists
\newcommand\SetSymbol[1][]{
   \nonscript\,#1\vert \allowbreak \nonscript\,\mathopen{}}
\DeclarePairedDelimiterX\Set[1]{\lbrace}{\rbrace}%
 { \renewcommand\given{\SetSymbol[\delimsize]} #1 }
\DeclarePairedDelimiterX{\Bkt}[1]{[}{]}{#1}
\DeclarePairedDelimiterX{\Paren}[1]{(}{)}{#1}
\DeclarePairedDelimiterX{\Abs}[1]{\lvert}{\rvert}{#1}

%%%%%%%%%%%%%%%%%%%%%%%%%%%%%%%%%%%%%%%%%%%%%%%%%%%

\newcolumntype{R}{>{\displaystyle}r}
\newcolumntype{C}{>{\displaystyle}c}
\newcolumntype{L}{>{\displaystyle}l}

%%%%%%%%%%%%%%%%%%%%%%%%%%%%%%%%%%%%%%%%%%%%%%%%%%%

\newcommand{\on}{\mathrm{on}}
\newcommand{\off}{\mathrm{off}}
\newcommand{\BOPS}{\mathrm{BOPS}}

%%%%%%%%%%%%%%%%%%%%%%%%%%%%%%%%%%%%%%%%%%%%%%%%%%%

\begin{document}
\title{Online-Offline Retailing Cooperation with BOPS Scheme under Price Competition}
\date{\today}
\author[1]{Chung-Chia Kuo\thanks{b05703100@ntu.edu.tw}}
\author[1]{Yu-Chieh Kuo\thanks{b07611039@ntu.edu.tw}}
\author[2]{Che Cheng\thanks{b07901012@ntu.edu.tw}}
\affil[ ]{(Team 4)}
\affil[1]{Department of Information Management, National Taiwan University}
\affil[2]{Department of Electrical Engineering, National Taiwan University}
\maketitle
\begin{abstract}
    Cooperation between competing agents are commonly seen in the retail industry,
    and a BOPS (buy online and pick up in-store) channel set up by an online retailer and a brick-and-mortar one is an example.
    This study formulates the aforementioned relationship using a game-theoretic model,
    and finds out that cooperation may not always succeed even  if both retailers may benefit from cooperation.
    In particular,
    cooperation is only possible if the BOPS channel is more efficient than the original offline channel.
    This finding holds regardless of the efficiency of online channel,
    consumers' willingness to pay and the traveling cost as long as all parameters are within the range in consideration.
\end{abstract}

\section{Introduction}
The retail industry has gone through huge evolution ever since the popularization of the Internet.
The rise of online retailing has affected the whole industry in all aspects,
among them channel selection, pricing and logistics.
While online retailing becomes of increasing importance,
brick-and-mortar business still plays a major role in our everyday life.

The relationship between online and offline retailers has been an interesting topic.
On the one hand,
different retailers compete in price, product quality, service, etc. for consumers,
and the Internet has definitely heated the competition.
On the other hand,
integration and cooperation between the two channels are seen in various aspects,
such as online-to-offline (O2O),
buy online and consume offline,
buy online and pick up in-store (BOPS), etc.
It is evident that both channels have non-substitutable strengths,
and it may be efficient for them to cooperates in certain situations.

In Taiwan,
online retailers,
such as Shopee, Momo, PChome, etc.,
offer an option for consumers to pick their orders up at convenience stores near their house.
Although providing various options for consumers may allow the online retailers to raise the price,
and cooperating with online retailers may bring more consumers into their stores,
the competition between both channels still exists,
and under what conditions may the cooperation work is an interesting topic.

Motivated by the aforementioned relationship,
we formulate the interaction with game-theoretic model as follows.
Consider an online retailer and a brick-and-mortar retailer.
The two retailers compete in price for an identical product.
In addition,
the brick-and-mortar retailer alone serves another product,
say coffee,
which the online retailer cannot deliver due to the short lifespan of the product.
We aims to answer when and how the two retailers may cooperate under the BOPS scheme.

Our major findings are summarized as follows.
First,
we find that cooperation has two major effect,
the efficiency effect and the collusion effect.
The efficiency effect is related to the relative efficiency between offline channel and BOPS channel,
and the collusion effect is that when both retailers cooperate,
they monopolize the market and can thus raise the price accordingly.
Both effects may increase the cooperating payoff.
Secondly,
and most importantly,
the cooperation may fail even if cooperating is beneficial,
as the brick-and-mortar retailer may deviate and earns even more given that the online retailer sets a high price due to collusion.
In particular,
the cooperation will succeed if the BOPS channel is more efficient than the offline channel.

The remainder of this paper is organized as follows.
\cref{sec:lit} reviews the literature.
\cref{sec:model} describes the model.
In \cref{sec:analysis},
we derive equilibrium and draw economic and managerial implications.
Finally,
\cref{sec:con} concludes the paper.
In addition,
all proofs are listed in the \hyperlink{sec:app}{Appendix}.

\section{Literature Review}\label{sec:lit}
Our study primarily relates to the topics of (1) coopetition between retailers,
and (2) online-offline relationship,
especially the adoption of BOPS.

Despite being in competition against each other,
retailers have been found to develop various cooperation modes under different scenarios.
In \cite{refer,refer_vert},
the authors discuss the coopetition between online retailers via in-store referral,
with retailer having horizontal \cite{refer} and vertical \cite{refer_vert} differences.
It is shown that the major tradeoff is between the effect of market expansion and price competition,
and cooperation is possible especially with third-party referral \cite{refer},
or when the difference in product qualities is large \cite{refer_vert}.
Also,
revenue sharing makes it possible for the two competing retailers to raise the price accordingly,
thus leveraging the competition.

As consumers are getting  more and more accustomed to shopping both online and offline,
interesting relationship between consumers and the two channels have been developed.
In \cite{consumer},
the authors investigate how consumer heterogeneity in purchase cost affects the online-offline competition,
and thus the equilibrium strategies of both retailers and consumers.
The authors of \cite{showroom} discuss the competition between two online retailers and an offline showroom with a signaling model.
The impact of BOPS adoption on a retailer is studied in \cite{BOPS_single},
where the channels online, offline, and BOPS are studied.
In \cite{BOPS_coop},
how online and offline retailers cooperate under the BOPS scheme is studied,
with revenue sharing, service subsidy, and inventory subsidy as channel coordination instruments.
In \cite{BOPS_dual},
the competition between two dual channel retailers is examined.
The two retailers decides whether or not to offer the BOPS channel,
and then compete in a Stackelberg game.

Our study complements the literature by discussing the cooperation of competing retailers under the BOPS scheme,
which is,
to the best of the authors' knowledge,
yet to be investigated.

\section{Model}\label{sec:model}
Consider a brick-and-mortar retailer (retailer R) and an online retailer (retailer O).
Retailer R sells through offline channel only,
while retailer O sells through online channel only originally.
Retailer R alone sells product A,
and the two retailers compete in price for an identical product B.
Retailer O now has a chance to offer an cooperation contract to set up a BOPS channel with retailer R together.
We assume that product A is less valuable and less costly in comparison with product B by assuming the unit cost of product A $C_A=0$ is less than that of product B for all channels,
and the consumers' willingness to buy for product A is $V_A=1$,
which is also less than that of product B,
i.e. $V>1$.
A detailed list of notations used in this paper is shown in \cref{tab:param}.

\begin{table}[h]
    \begin{tabularx}{\linewidth}{>{$}c<{$}X}
            \toprule
        \off ,\on ,\BOPS & Subscript index of the three channels, $\off$ denotes offline channel, $\on$ denotes online channel, and $\BOPS$ denotes BOPS channel. \\
        N,C              & Superscript index of the two scenarios, $N$ denotes no cooperation, and $C$ denotes cooperation.                                       \\
        D                & Superscript index of retailer R deviating in cooperation.                                                                              \\
        C_{\off}         & Unit cost of product B sold through offline channel for retailer R.                                                                         \\
        C_{\on}          & Unit cost of product B sold through online channel for retailer O.                                                                          \\
        C_{\BOPS}        & Unit cost of product B sold through BOPS channel for retailer O in case of cooperation.                                                     \\
        V                & Utility a consumer derives after getting product B.                                                                                    \\
        T                & Additional cost a consumer incurs when buying anything through offline or BOPS channel.                                                                \\
        x                & Type-dependent additional cost a consumer incurs when buying anything through online channel.                                                          \\
        p_A              & Unit retail price for product A sold through offline channel set by retailer R.                                                             \\
        p_{\off}         & Unit retail price for product B sold through offline channel set by retailer R.                                                             \\
        p_{\on}          & Unit retail price for product B sold through online channel set by retailer O.                                                              \\
        p_{\BOPS}        & Unit retail price for product B sold through BOPS channel set by retailer O in case of cooperation.                                         \\
        u                & Lump sum payment from retailer O to retailer R in case of cooperation.                                                                 \\
        v                & Unit fee paid by retailer O to retailer R for each product B sold through BOPS channel in case of cooperation.                              \\
        D_A              & Demand for product A in offline channel.                                                                                               \\
        D_{\off}         & Demand for product B in offline channel.                                                                                               \\
        D_{\on}          & Demand for product B in online channel.                                                                                                \\
        D_{\BOPS}        & Demand for product B in BOPS channel in case of cooperation.                                                                           \\
        \Pi_R            & Payoff for retailer R.                                                                                                                 \\
        \Pi_O            & Payoff for retailer O.                                                                                                                 \\
        \eta             & Increase in total payoff for both retailers after cooperation.                                                                         \\
        \bottomrule
    \end{tabularx}
    \caption{Summary of Important Notations}
    \label{tab:param}
\end{table}

The consumers are heterogeneous in the disutility of product being delivered to their house,
and they are assumed to distribute in $[0,1]$ uniformly with density 1.
All consumers either buy 1 unit of both products,
1 unit of one of the products,
or nothing at all.
A consumer endowed with type $x$ (consumer $x$) derives an utility of 1 and $V\geq 1$ upon getting product A and B respectively,
and incurs a cost $T>0$ if he buys anything through offline or BOPS channel.
He incurs a mental cost $x$ for buying through the online channel.
In case that a consumer buys nothing at all,
his utility is 0.

For retailer R,
selling one unit of product A at price $p_A$ generates a profit of $p_A$.
\footnote{We normalize the unit cost of product A to 0.
In particular,
we assume product A is both},
and selling one unit of product B at price $p_{\off}$ generates a profit of $p_{\off}-C_{\off}$.
For retailer O,
selling one unit of product B at price $p_{\on}$ generates a profit of $p_{\on}-C_{\on}$,
and in case of cooperation,
selling one unit of product B at price $p_{\BOPS}$ using BOPS channel generates a profit of $p_{\BOPS}-C_{\BOPS}-v$%
\footnote{As there's no information asymmetry in the model,
    we assume that all cost are taken by retailer O for simplicity,
    as it only cause a constant shift in the derived $u$ and is thus of little significance.}.

The game proceeds as follows.
First,
retailer O offers a contract $(u,v)$ to retailer R,
and retailer R decides whether to take it or not.
In case he accepts the contract,
the two retailers then decide simultaneously $p_{\on},p_{\BOPS}$ and $p_A,p_{\off}$ respectively.
If he rejects the contract,
then the two retailers decide simultaneously $p_{\on}$ and $p_A,p_{\off}$.
Finally,
each consumer chooses the optimal consumption combination,
and the revenues are realized.

To focus on the situation of interest,
we made the following assumptions
\begin{enumerate}
    \item $C_{\off},C_{\on},C_{\BOPS}$ are sufficiently small,
          such that selling through any channel is in principle efficient.
    \item $1\leq T\leq V$,
          i.e. consumers will only buy product A if they are buying product B through offline or BOPS channel,
          and traveling to the store to buy product B may be beneficial for the consumers.
\end{enumerate}
The exact conditions needed for the first assumption will be derived in \cref{sec:analysis}.
Although the second assumption may seem different from real-life experience,
it captures the main assumption that traveling to the shop is costly.

\section{Analysis}\label{sec:analysis}
To ease notation,
we denote the action of consumers as sets,
e.g. $\Set*{A,B_{\on}}$ denotes the action of buying product A and buying product B through online channel.

\subsection{Equilibrium Analysis}\label{subsec:eq}
We start our analysis from the benchmark scenario,
where no cooperation takes place.
First observe that for each consumer,
$\Set*{A}$ is dominated by $\varnothing$,
and $\Set*{A, B_{\on}}$ is dominated by $\Set*{B_{\on}}$.
Also,
$\Set*{A,B_{\off}}$ dominates $\Set*{B_{\off}}$ if and only if $p_A\leq 1$,
which must be the case in equilibrium,
as selling A is always profitable.
Next observe that $\Set*{A,B_{\off}}$ dominates $\varnothing$ if and only if $p_A+p_{\off}\leq 1+V-T$,
which must also be the case in equilibrium,
since retailer R earns nothing otherwise.

Thus,
each consumer chooses the better option between $\Set*{A,B_{\off}}$ and $\Set*{B_{\on}}$,
and the utility of consumer $x$ choosing these two options are $u_{\Set*{A,B_{\off}}}=1+V-T-(p_A+p_{\off})$ and $u_{\Set*{B_{\on}}}=V-x-p_{\on}$ respectively.
The indifferent consumer is that of type $x^N=-1+T-p_{\on}+(p_A+p_{\off})$,
and thus the demands are $D_A=D_{\off}=1-x^N=2-T+p_{\on}-(p_A+p_{\off})$ and $D_{\on}=x^N=-1+T-p_{\on}+(p_A+p_{\off})$.

After some simple derivations,
it can be seen that the equilibrium prices $p_R^N\equiv p_A^N+p_{\off}^N$ and $p_{\on}^N$ and equilibrium profits $\Pi_R^N$ and $\Pi_O^N$ are
\[
    p_R^N=1+\frac{-T+2C_{\off}+C_{\on}}{3}\,\text{, } p_{\on}^N=\frac{T+C_{\off}+2C_{\on}}{3}\,,
\]
\[
    \Pi_R^N=\Paren*{1-\frac{T+C_{\off}-C_{\on}}{3}}^2\,,\text{ and }\Pi_O^N=\Paren*{\frac{T+C_{\off}-C_{\on}}{3}}^2\,.
\]
Note that $p_A^N$ can be chosen arbitrarily in $[0,1]$,
and the constraint $p_A^N+p_{\off}^N\leq 1+V-T$ is satisfied whenever $2T\leq 3V-2C_{\off}-C_{\on}$,
which we assume is true for simplicity.
We also assume $0\leq T+C_{\off}-C{\on}\leq 3$,
so that both demands are nonnegative.

It can be seen that in the current scenario,
the effect of price competition dominates the pricing strategy,
as the willingness to pay $V$ is of no effect.

In case of cooperation,
the two retailers set up a BOPS channel together.
The consumers now have extra options $\Set*{B_{\BOPS}}$ and $\Set*{A,B_{\BOPS}}$.
Once again,
we have $\Set*{A,B_{\BOPS}}$ dominates $\Set*{B_{\BOPS}}$.
Also,
note that $\Set*{A,B_{\BOPS}}$ dominates $\Set*{A,B_{\off}}$ if and only if $p_{\off}\geq p_{B,BOPS}$,
and similarly,
$\Set*{A,B_{\BOPS}}$ dominates $\varnothing$ if and only if $p_A+p_{\BOPS}\leq 1+V-T$,
which must both be the case in equilibrium,
since the cooperation would be pointless if not a single unit is sold using the BOPS channel.
We assume $p_{\off}\geq p_{\BOPS}$ and ignore $p_{\off}$ here for simplicity,
as it does not affect the equilibrium profits.
Whether R will deviate and choose $p_{\off}<p_{BOPS}^C$ is later discussed in \cref{subsec:IC}.

Each consumer thus chooses the better option between $\Set*{A,B_{\BOPS}}$ and $\Set*{B_{\on}}$,
and the utility of consumer $x$ choosing these two options are $u_{\Set*{A,B_{\BOPS}}}=1+V-T-p_A-p_{\BOPS}$ and $u_{\Set*{B_{\on}}}=V-x-p_{\on}$ respectively.
The indifferent consumer is that of type $x^C=-1+T-(p_{\on}-p_{\BOPS})+p_A$,
and the demands are $D_A=D_{\BOPS}=1-x^C=2-T-p_A+(p_{\on}-p_{\BOPS})$,
$D_{\on}=x^C=-1+T+p_A-(p_{\on}-p_{\BOPS})$,
and $D_{\off}=0$.

After rewriting $(p_{\on}-p_{\BOPS})$ as $p$,
retailer R now solves
\[
    \begin{array}{RL}
        \max_{p_A}  & (p_A+v)\Paren*{2-T-p_A+p}+u \\
        \text{s.t.} & p_A+p_{\BOPS}\leq 1+V-T\,.
    \end{array}
\]
and retailer O solves
\[
    \begin{array}{RL}
        \max_{p,p_{\BOPS}} & (p+v+C_{\BOPS}-C_{\on})\Paren*{-p-1+T+p_A}+p_{\BOPS}-C_{\BOPS}-v-u \\
        \text{s.t.}        & p_A+p_{\BOPS}\leq 1+V-T\,.
    \end{array}
\]

Through simple derivations,
it can be shown that the equilibrium prices $p_A^C$, $p_{\on}^C$, and $p_{\BOPS}^C$ and equilibrium profits $\Pi_R^C$ and $\Pi_O^C$ are
\[
    p_A^C=1-v-\frac{T+C_{\BOPS}-C_{\on}}{3}\,\text{, } p_{\on}^C=V-\frac{T+C_{\BOPS}-C_{\on}}{3}\,\text{, } p_{\BOPS}^C=V+v-T+\frac{T+C_{\BOPS}-C_{\on}}{3}\,,
\]
\[
    \Pi_R^C=\Paren*{1-\frac{T+C_{\BOPS}-C_{\on}}{3}}^2+u\,,\text{ and }\Pi_O^C=\Paren*{\frac{T+C_{\BOPS}-C_{\on}}{3}}^2-\frac{2T+2C_{\BOPS}+C_{\on}}{3}+V-u\,.
\]
Again,
we assume $0\leq T+C_{\BOPS}-C_{\on}\leq 3$ such that both demands are nonnegative.
Also,
the choice of $v$ must satisfies $-T-C_{\BOPS}+C_{\on}\leq 3v\leq 3-T-C_{\BOPS}+C_{\on}$ so that $0\leq p_A^C\leq 1$.

\begin{proposition}\label{prop:v}
    The unit fee $v$ has no effect on the equilibrium payoff of both retailers and all consumers when cooperation works.
\end{proposition}

\cref{prop:v} shows that the choice of $v$ has no effect on the equilibrium utility of all players.
In particular,
when $v$ is raised by $\Delta v$,
then $p_A^C$ is decreased by $\Delta v$ and $p_{\BOPS}^C$ is increased by $\Delta v$,
and thus the consumers choosing $\Set*{A,B_{\BOPS}}$ still face the total price $V+1-T$.
This being said,
we still cannot conclude that $v$ is redundant.
Its effect is discussed in \cref{subsec:IC}.
Also,
we believe that such result occurs mainly because the BOPS channel substitutes the offline channel completely in the current setting,
and that it may not hold in more general situations,
e.g. consumers are heterogeneous in traveling cost, etc.

\subsection{Efficiency Analysis}\label{subsec:eff}
For the cooperation to take place,
the two retailers together must earn more than before cooperation,
i.e. $\Pi_R^C+\Pi_O^C\geq \Pi_R^N+\Pi_O^N$.
To ease notation,
we define the efficiency $\eta\equiv \Pi_R^C+\Pi_O^C-\Pi_R^N-\Pi_O^N$.

\begin{proposition}\label{prop:eff_C}
    $\eta\geq 0$ if $C_{\off}\geq C_{\BOPS}$.
\end{proposition}

\cref{prop:eff_C} shows a sufficient condition for cooperation to be efficient.
In fact,
it captures the efficiency improvement of substituting offline channel with online channel,
which does not harm the consumers.
The converse is,
however,
not true in general,
and a counterexample can be constructed easily using \cref{prop:eff_V}.

\begin{proposition}\label{prop:eff_V}
    Given $T,C_{\off},C_{\on},C_{\BOPS}$,
    there exists some sufficiently large $V$ such that $\eta\geq 0$ holds.
\end{proposition}

\cref{prop:eff_V} captures the collusion effect of the cooperation.
In effect,
when the two channel cooperates,
retailer O monopolizes the market for product B,
and can thus raise the price according to the consumers' willingness to pay $V$.
In contrast,
in the no-cooperation scenario,
the price competition makes the equilibrium price independent of $V$ and is set according to $C_{\off}$, $C_{\on}$, and $T$.
As all consumers are served in both scenarios,
this improvement only benefits the two retailers,
and harms the consumers.

\subsection{Incentive Compatibility Analysis}\label{subsec:IC}
The conditions discussed in \cref{subsec:eff} is not sufficient for cooperation to take place.
After the contract is signed,
retailer R may choose to deviate by choosing some $p_{\off}<p_{\BOPS}^C$ and some $p_A$ accordingly.
Such deviation will happen if and only if deviating earns him more than cooperating.
By plugging $p_{\on}^C$ into the best response of retailer A in no-cooperation scenario,
we obtain the optimal deviation price $p_R^D\equiv p_A^D+p_{\off}^D$ and the corresponding payoff $\Pi_R^D$,
which are
\[
    p_R^D=1+\frac{3V-4T+3C_{\off}+C_{\on}-C_{\BOPS}}{6}\,\text{, and }\Pi_R^D=\Paren*{1+\frac{3V-4T-3C_{\off}+C_{\on}-C_{\BOPS}}{6}}^2+u\,.
\]
For the deviation to succeed,
it must be the case that $p_{\off}^D<p_{\BOPS}^C$,
which is most likely when retailer A chooses $p_A^D=1$.

\begin{proposition}\label{prop:dev_v}
    $p_{\off}^D<p_{\BOPS}^C$ if and only if $C_{\off}+C_{\on}-C_{\BOPS}<V+2v$.
\end{proposition}

\cref{prop:dev_v} shows an interesting result.
Instead of choosing a larger $v$,
it is better to choose a smallest possible $v$ to prevent retailer R from deviating.
Recall that $v$ is constrained in \cref{subsec:eq},
and the lower bound is $v^*=-\tfrac{T+C_{\BOPS}-C_{\on}}{3}$.
Plugging $v^*$ into \cref{prop:dev_v} gives the following corollary:

\begin{corollary}
    Deviation may succeed if and only if $3C_{\off}<3V-2T+C_{\BOPS}-C_{\on}$.
\end{corollary}

We next analyze when retailer R would want to deviate.

\begin{proposition}\label{prop:dev_pi}
    $\Pi_R^D>\Pi_R^C$ if and only if $3C_{\off}<3V-2T+C_{\BOPS}-C_{\on}$.
\end{proposition}

\cref{prop:dev_pi} gives a condition for when deviating is beneficial,
which coincides with that in \cref{prop:dev_v}.
Thus,
we obtain the following remark:

\begin{remark}
    Retailer R would like to deviate if and only if deviation is possible.
\end{remark}

This also suggests that $v$ is indeed redundant for the model,
since the optimal $v$ can still only prevent retailer R from deviating when he does not want to deviate.

By applying the assumption that $3V\geq 2T+2C_{\off}+C_{\on}$ on \cref{prop:dev_pi},
we obtain the following corollary:

\begin{corollary}\label{cor:dev_C}
    $\Pi_R^D>\Pi_R^C$ if $C_{\BOPS}>C_{\off}$.
\end{corollary}

This shows that the potential collusion under large $V$ as discussed in \cref{subsec:eff} cannot be realized due to moral hazard,
as retailer R would implement a smaller $p_R$ and take over the demand.
We may thus conclude that cooperation will take place if and only if both $3C_{\off}\geq 3V-2T+C_{\BOPS}-C_{\on}$ and $\eta\geq 0$ holds,
and the former implies $C_{\off}\geq C_{\BOPS}$ (\cref{cor:dev_C}) and thus implies the latter (\cref{prop:eff_C}).
Therefore,
we obtain the following result:

\begin{corollary}\label{cor:coop}
    Cooperation takes place if and only if $C_{\off}\geq C_{\BOPS}$.
\end{corollary}

The result shows that the attempt of collusion would fail if the BOPS channel is less efficient than the offline channel.
On the other hand,
if the BOPS channel is more efficient,
then the two retailers may cooperate without worrying anyone would deviate,
and benefit from both both the efficiency effect and the collusion effect brought by cooperation.

In reality,
whether retailer R drops out from the offline channel or not nay seem contractible,
and thus the potential deviation in this model may be resolved.
However,
we believe that the moral hazard problem would always be present in such coopetition,
and it may in general be impossible to verify every detail in more complex situations.
Thus,
although not precisely,
the simplified model does capture the nature of the real-world situation.

\section{Conclusions}\label{sec:con}
This paper formulates an analytic model to study the cooperation between competing online and brick-and-mortar retailers to implement a BOPS channel.
By discussing the efficiency and incentive compatibility issues,
we derive the conditions for successful cooperation.
Our study shows that cooperation may benefit the two retailers in two ways.
First,
if the BOPS channel is more efficient than the offline channel,
then cooperation may increase the total profit by reducing the unit cost.
Secondly,
through cooperation,
the two retailers may collude and alleviate the effect of price competition.
However,
our result also indicates that collusion fails when the offline channel is more efficient than the BOPS channel.
Such failure occurs because the brick-and-mortar retailer would want to deviate and lower its price.

As with any research,
this study also has several limitations.
First,
the model oversimplifies the scenario,
thus producing a few unconvincing results,
such as the perfect substitution between offline and BOPS channel.
This may potentially be improved by adding more heterogeneity into the consumer,
such as those discussed in \cite{consumer}.
Secondly,
the model considers only two retailers,
thus cooperation leads directly to collusion.
Finally,
we focus only on pure-strategy equilibria throughout the paper.
We believe that cooperation may be possible even if the BOPS channel is less efficient if mixed-strategy is considered for pricing.

\section*{\hypertarget{sec:app}{Appendix}}
\addcontentsline{toc}{section}{Appendix}
\begin{proof}[Proof of \cref{prop:v}]
    For consumers choosing $\Set*{B_{\on}}$,
    $\frac{\partial p_{\on}^C}{\partial v}=0$,
    so he is not affected by $v$.
    For consumers choosing $\Set*{A,B_{\BOPS}}$,
    $\frac{\partial (p_A^C+p_{\BOPS}^C)}{\partial v}=0$,
    so he is not affected,
    and no consumer will deviate to different options.

    For retailer R and retailer O,
    $\frac{\partial \Pi_R^C}{\partial v}=\frac{\partial \Pi_O^C}{\partial v}=0$,
    so they are not affected either.
\end{proof}

\begin{proof}[Proof of \cref{prop:eff_C}]
    Given that $C_{\off}\geq C_{\BOPS}$,
    we have
    \[
        \begin{array}{RCL}
            \eta & = & \Paren*{1-\frac{T+C_{\BOPS}-C_{\on}}{3}}^2-\Paren*{1-\frac{T+C_{\off}-C_{\on}}{3}}^2 \\[3mm]
                 &   & +\Paren*{\frac{T+C_{\BOPS}-C_{\on}}{3}}^2-\Paren*{\frac{T+C_{\off}-C_{\on}}{3}}^2    \\[3mm]
                 &   & -\frac{2T+2C_{\BOPS}+C_{\on}}{3} +V                                                  \\[3mm]
                 & = & \frac{1}{9}\biggl((C_{\off}-C_{\BOPS})(6-4T-2C_{\BOPS}-2C_{\off}+4C_{\on})           \\[3mm]
                 &   & \;+9V-6T-6C_{\BOPS}-3C_{\on}\biggr)\,.
        \end{array}
    \]
    Using the assumption that $C_{\on}-C_{\off}\geq T-3$ and $C_{\on}-C_{\BOPS}\geq T-3$ from \cref{subsec:eq},
    we have
    \[
        \begin{array}{RCL}
            \eta & \geq & \frac{1}{9}\biggl((C_{\off}-C_{\BOPS})(6-4T+4T-12)+9V-6T-6C_{\BOPS}-3C_{\on}\biggr) \\[3mm]
                 & =    & \frac{1}{9}(-6C_{\off}+9V-6T-3C_{\on})\,.
        \end{array}
    \]
    Using the assumption that $3V\geq 2T+2C_{\off}+C_{\on}$ from \cref{subsec:eq},
    we have
    \[
        \begin{array}{RCL}
            \eta & \geq & \frac{1}{9}(-6C_{\off}+9V-6T-3C_{\on})                    \\[3mm]
                 & \geq & \frac{1}{9}(-6C_{\off}+6T+6C_{\off}+3C_{\on}-6T-3C_{\on}) \\[3mm]
                 & =    & 0\,.
        \end{array}
    \]
\end{proof}

\begin{proof}[Proof of \cref{prop:eff_V}]
    Note that $\frac{\partial \eta}{\partial V}=1$,
    and $V$ is not bounded above by any assumption,
    thus $\eta$ can be arbitrarily large given a large $V$.
\end{proof}

\begin{proof}[Proof of \cref{prop:dev_v}]
    Observe that
    \[
        \begin{array}{RRCL}
                 & p_{\off}^D                                  & < & p_{\BOPS}^C                         \\[3mm]
            \iff & \frac{3V-4T+3C_{\off}+C_{\on}-C_{\BOPS}}{6} & < & V+v-T+\frac{T+C_{\BOPS}-C_{\on}}{3} \\[3mm]
            \iff & 3V-4T+3C_{\off}+C_{\on}-C_{\BOPS}           & < & 6V+6v-6T+2T+2C_{\BOPS}-2C_{\on}     \\[3mm]
            \iff & C_{\off}+C_{\on}-C_{\BOPS}                  & < & V+2v\,.
        \end{array}
    \]
\end{proof}

\begin{proof}[Proof of \cref{prop:dev_pi}]
    Observe that
    \[
        \begin{array}{RCL}
            \Pi_R^D-\Pi_R^C & = & \Paren*{1+\frac{3V-4T-3C_{\off}+C_{\on}-C_{\BOPS}}{6}}^2-\Paren*{1-\frac{T+C_{\BOPS}-C_{\on}}{3}}^2         \\[3mm]
                            & = & \Paren*{1+\frac{3V-4T-3C_{\off}+C_{\on}-C_{\BOPS}}{6}+1-\frac{T+C_{\BOPS}-C_{\on}}{3}}\cdot                 \\[3mm]
                            &   & \Paren*{\Paren*{1+\frac{3V-4T-3C_{\off}+C_{\on}-C_{\BOPS}}{6}}-\Paren*{1-\frac{T+C_{\BOPS}-C_{\on}}{3}}}    \\[3mm]
                            & = & \Paren*{2+\frac{V-2T+C_{\on}-C_{\BOPS}-C_{\off}}{2}}\Paren*{\frac{3V-2T+C_{\BOPS}-C_{\on}-3C_{\off}}{6}}\,.
        \end{array}
    \]
    We shall first focus on the first term.
    Note that using the assumptions $C_{\on}-C_{\off}\geq T-3$, $C_{\on}-C_{\BOPS}\geq T-3$ and $3V\geq 2T+2C_{\off}+C_{\on}$ from \cref{subsec:eq},
    we have
    \[
        \begin{array}{RCL}
            2+\frac{V-2T+C_{\on}-C_{\BOPS}-C_{\off}}{2} & \geq & 2+\frac{V-2T+T-3-C_{\off}}{2}                 \\[3mm]
                                                        & =    & \frac{3+3V-3T-3C_{\off}}{6}                   \\[3mm]
                                                        & \geq & \frac{3+2T+2C_{\off}+C_{\on}-3T-3C_{\off}}{6} \\[3mm]
                                                        & =    & \frac{3-T+C_{\on}-C_{\off}}{6}                \\[3mm]
                                                        & \geq & \frac{3-T+T-3}{6}                             \\[3mm]
                                                        & =    & 0\,.
        \end{array}
    \]
    Therefore,
    $\Pi_R^D-\Pi_R^C$ is nonnegative if and only if the second term is nonnegative,
    i.e. $3C_{\off}<3V-2T+C_{\BOPS}-C_{\on}$.
\end{proof}


\bibliographystyle{unsrt}
\bibliography{ref}
\addcontentsline{toc}{section}{References}

\end{document}