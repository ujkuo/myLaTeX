%%%%%%%%%%%%%%%%%%%%%%%%%%%%%%%%%%%%%%%%%%%%%%%%%%%%%%%%%%%%%%%
\PassOptionsToPackage{subsection=false}{beamerouterthememiniframes} % delete subsection line
\documentclass{beamer}
\usepackage{tikz}
\usepackage{array}
\usepackage{url}
\usepackage{amsmath, amsthm, amsfonts, amssymb, bm}
\usepackage{booktabs}
\usepackage{enumerate}
\usepackage[shortlabels]{enumitem}
\usepackage{pifont}
\usepackage{framed}
\usepackage{textcomp}
\usepackage{caption}
\usepackage{subcaption}
\usepackage{xeCJK}
\usepackage[natural, svgnames]{xcolor}
\usepackage{natbib}
\bibliographystyle{econ}
\usepackage{hyperref}
\hypersetup{
    colorlinks,
    citecolor=myblue,
    linkcolor=myblue,
    urlcolor=myred 
}


\usepackage[mathscr]{eucal}
\usepackage{tcolorbox}

%%%%%%%%%%%%%%%%%%%%%%%%%%%%%%%%%%%%%%%%%%%%%%%%%%%%%%%%%%%%%%%
% beamer theme setting
\usetheme{Szeged}
\usecolortheme{rose}
\usefonttheme{serif}

%%%%%%%%%%%%%%%%%%%%%%%%%%%%%%%%%%%%%%%%%%%%%%%%%%%%%%%%%%%%%%%
% Unable the control buttons at bottom
\setbeamertemplate{navigation symbols}{}

%%%%%%%%%%%%%%%%%%%%%%%%%%%%%%%%%%%%%%%%%%%%%%%%%%%%%%%%%%%%%%%
% Define my color
\definecolor{myred}{RGB}{192,46,46}
\definecolor{myblue}{RGB}{7,75,164}
\definecolor{mygreen}{rgb}{0.11,0.7,0.02}
\definecolor{myyellow}{rgb}{0.7,0.625,0.1}

%%%%%%%%%%%%%%%%%%%%%%%%%%%%%%%%%%%%%%%%%%%%%%%%%%%%%%%%%%%%%%%
% Modify footer
\newcommand{\shortauthor}{\textcolor{myblue}{Yu-Chieh Kuo}}
\setbeamerfont{title in head/foot}{family=\chancery}
\setbeamertemplate{footline}
{%
  \begin{beamercolorbox}[colsep=1pt]
      {upper separation line foot}
  \end{beamercolorbox}
  \hbox{%
      \begin{beamercolorbox}[wd=0.333333\paperwidth, ht=4ex, 
          dp=2.3ex, leftskip=.3cm plus1fill, rightskip=.3cm]
          {title in head/foot}%
          \usebeamerfont{title in head/foot}
          \parbox{.28\paperwidth}{\raggedright\inserttitle}
    \end{beamercolorbox}%
    \begin{beamercolorbox}[wd=0.333333\paperwidth, ht=4ex, 
        dp=2ex, center]{title in head/foot}%
      \usebeamerfont{author in head/foot}
      \insertframenumber/\inserttotalframenumber
    \end{beamercolorbox}%
    \begin{beamercolorbox}[wd=0.333333\paperwidth, ht=4ex,
        dp=2ex, leftskip=.3cm, rightskip=.3cm plus1fil]
        {title in head/foot}%
        \usebeamerfont{title in head/foot}
        \parbox{.28\paperwidth}{\raggedleft\shortauthor}
    \end{beamercolorbox}}
  \begin{beamercolorbox}[colsep=1pt]
      {lower separation line foot}
  \end{beamercolorbox}
}

\setbeamertemplate{headline}
{
    \begin{beamercolorbox}[colsep=1pt]
        {upper separation line head}
    \end{beamercolorbox}
    \hbox{
        \begin{beamercolorbox}[ignorebg,ht=2.25ex,dp=3.75ex]{section in head/foot}
                \insertnavigation{\paperwidth}
            \end{beamercolorbox}%:w

    }
    \begin{beamercolorbox}[colsep=1pt]
        {lower separation line head}
    \end{beamercolorbox}
}

%%%%%%%%%%%%%%%%%%%%%%%%%%%%%%%%%%%%%%%%%%%%%%%%%%%%%%%%%%%%%%%
% Modify theorem and block
\usepackage{parskip}
\usepackage{microtype}
\newlength{\prevparskip}

\defbeamertemplate{block begin}{lines}{%
    \par\vskip2ex plus 1fil%
    \setlength{\prevparskip}{\parskip}%
    \setlength{\parskip}{0pt}%
    \begin{beamercolorbox}{block title}%
        \usebeamerfont{block title}%
        \vskip-2ex%
        \insertblocktitle%
        \vskip-2ex%
        {\color{myred}\rule{\textwidth}{1pt}}%
        \vskip 1ex%
    \end{beamercolorbox}%
    \nointerlineskip%
    \begin{beamercolorbox}[vmode]{block body}%
        \usebeamerfont{block body}%
        %\vskip 1.2ex%
}

\defbeamertemplate{block end}{lines}{%
    \end{beamercolorbox}%
    \vskip-1ex%
    \color{myblue}\rule{\textwidth}{1pt}%
    \setlength{\parskip}{\prevparskip}%
    \vskip-1ex plus 1fil
}

\setbeamertemplate{blocks}[lines]

%%%%%%%%%%%%%%%%%%%%%%%%%%%%%%%%%%%%%%%%%%%%%%%%%%%%%%%%%%%%%%%
% Theorem, lemma, proposition
% Numbering from 1 for each environments
\setbeamertemplate{theorems}[numbered] 
\newtheorem{coro}{Corollary}
%\newtheorem{lemma}{Lemma}
\newtheorem{propo}{Proposition}

%%%%%%%%%%%%%%%%%%%%%%%%%%%%%%%%%%%%%%%%%%%%%%%%%%%%%%%%%%%%%%%
% Modify section and subsection's entry
\setbeamertemplate{section in toc}{
  {\color{myblue}\inserttocsectionnumber.}~\inserttocsection}
\setbeamertemplate{subsection in toc}{
\hspace{1.2em}{\color{myblue}$\triangleright$}~\inserttocsubsection\par}

% Modify itemize's entry and position
\setlist[itemize]{label=\textcolor{myblue}{$\triangleright$}}
\setlength{\leftmargini}{0pt}

%%%%%%%%%%%%%%%%%%%%%%%%%%%%%%%%%%%%%%%%%%%%%%%%%%%%%%%%%%%%%%%
% Modify color

\setbeamercolor{block title}{fg=myblue, bg=white}
\setbeamercolor{block body}{fg=black, bg=white}
\setbeamercolor{block title alerted}{use=structure, fg=white, bg=myred!90!white}
\setbeamercolor{block body alerted}{use=structure, fg=black, bg=myred!20!white}
\setbeamercolor{frametitle}{fg=myblue}
\setbeamercolor{separation line}{bg=myblue}
\setbeamercolor{section in head/foot}{fg=myblue}
\setbeamercolor{subsectiopn in head/foot}{fg=myblue}
\setbeamercolor{section in toc}{fg=myblue}
\setbeamercolor{subsection in toc}{fg=myblue}
\setbeamercolor*{title}{fg=myblue}
\setlist[description]{format=\textcolor{myblue}}

%%%%%%%%%%%%%%%%%%%%%%%%%%%%%%%%%%%%%%%%%%%%%%%%%%%%%%%%%%%%%%%
% Modify margin setting

\justifying
\setbeamersize{text margin left=1.8em,text margin right=1.8em}
\setlength{\parindent}{2em}

\addtobeamertemplate{frametitle}{\setlength{\parindent}{0em}}{}
\addtobeamertemplate{block begin}{\setlength{\parindent}{0em}}{\setlength{\parindent}{2em}}
\addtobeamertemplate{block example begin}{\setlength{\parindent}{0em}}{\setlength{\parindent}{2em}}
\settowidth{\leftmargini}{\usebeamertemplate{itemize item}}
%\addtolength{\leftmargini}{\labelsep}

%%%%%%%%%%%%%%%%%%%%%%%%%%%%%%%%%%%%%%%%%%%%%%%%%%%%%%%%%%%%%%%
% Modify font

\setCJKmainfont{思源宋體 TW}
\setmainfont{Philosopher}
\setbeamerfont{title}{size=\huge, series=\bfseries}
\newfontfamily\chancery{Apple Chancery}
\DeclareTextFontCommand{\textch}{\chancery}
\setbeamerfont{block title}{family=\chancery}

%%%%%%%%%%%%%%%%%%%%%%%%%%%%%%%%%%%%%%%%%%%%%%%%%%%%%%%%%%%%%%%

\newcolumntype{R}{>{\displaystyle}r}
\newcolumntype{C}{>{\displaystyle}c}
\newcolumntype{L}{>{\displaystyle}l}

%%%%%%%%%%%%%%%%%%%%%%%%%%%%%%%%%%%%%%%%%%%%%%%%%%%%%%%%%%%%%%%
% Title page

\title{
    Who's Who in the Network.\\ Wanted: Key Player
}  
\author{郭宇杰\inst{1}, Yu-Chieh Kuo\inst{2}} 
\date{\today} 

\institute[NTU]
{
    \inst{1}
    Department of Information Management,
    National Taiwan University \\
    \inst{2}
    Department of Electronic Engineering,
    National Taiwan University
}

%%%%%%%%%%%%%%%%%%%%%%%%%%%%%%%%%%%%%%%%%%%%%%%%%%%%%%%%%%%%%%%
% Set frame numbering and hide subsecs in toc

\AtBeginSection[]
{
    \begin{frame}[noframenumbering]
      \frametitle{}
      \tableofcontents[currentsection, hideallsubsections]
    \end{frame}
}

%%%%%%%%%%%%%%%%%%%%%%%%%%%%%%%%%%%%%%%%%%%%%%%%%%%%%%%%%%%%%%%

\newcommand{\hl}[1]{\textcolor{myblue}{#1}}
\newcommand{\rhl}[1]{\textcolor{myred}{#1}}
\newcommand{\Pp}[1]{\text{Proposition #1}}
\newcommand{\Int}{\text{Intuition}}

%%%%%%%%%%%%%%%%%%%%%%%%%%%%%%%%%%%%%%%%%%%%%%%%%%%%%%%%%%%%%%%

%includeonlyframes{current}
\begin{document}

\begin{frame}
\titlepage
\end{frame} 

\begin{frame}[label=current]
    \frametitle{Motivation}
    \framesubtitle{}
    \begin{itemize}
        \item As a decision-maker or policymaker, we may want to find the most
           influential player in the network to break or strengthen such effect.
        \item As a decision-maker or policymaker, we may want to find the most
           influential player in the network to break or strengthen such effect.
        \item As a decision-maker or policymaker, we may want to find the most
           influential player in the network to break or strengthen such effect.
        \item As a decision-maker or policymaker, we may want to find the most
           influential player in the network to break or strengthen such effect.
    \end{itemize}
\end{frame}

\begin{frame}[label=current]
    \frametitle{Literature Review}
    \framesubtitle{}
    \begin{itemize}
        \item
    \end{itemize}
\end{frame}

\begin{frame}
    \frametitle{Outline}
    \tableofcontents
\end{frame} 

\section{Model Settings} 
\begin{frame}[label=current]
    \frametitle{Nash-Bonacich Equilibrium}
    \framesubtitle{}
    \begin{theorem}
        Let $\mu_1(\bm{G})$ be the largest eigenvalue of $\bm{G}$,
        %\footnote{$\mu_1(\bm{G})$ is well-define and larger than $0$ since all eigenvalues
        %of a symmetric matrix $\bm{G}$ are real, and the diagnal of $\bm{G}$ is zero.}
        the matrix $\beta[\bm{I}-\lambda^*\bm{G}]$ is well-defined and nonnegative
        if and only if $\beta>\lambda\mu_1(\bm{G})$, thus the unique interior Nash equilibrium
        is given by $\bm{x}^*(\bm{\Sigma})=\frac{\alpha}{\beta+\gamma b(\mathscr{G},\lambda^*)}
        b(\mathscr{G},\lambda^*)$.
    \end{theorem}
    \begin{itemize}
                   \setlength\itemsep{-0.5em}
        \item Given the unique Nash equilibrium 
            $\bm{x}^*(\bm{\Sigma})=\frac{\alpha}{\beta+\gamma b(\mathscr{G},\lambda^*)}
            b(\mathscr{G},\lambda^*)$, we want to analyze how three different effects influence
           the equilibrium.
           \begin{itemize}
                   \setlength\itemsep{-0.5em}
               \item There exists no equilibrium if the matrix of cross-effects $\bm{\Sigma}$ 
                   reduces to $\lambda\bm{G}$.
               \item There is a unique equilirium if $\bm{\Sigma}$ reduces to $-\beta\bm{I}-\gamma\bm{U}$.
           \end{itemize}
    \end{itemize}
\end{frame}

\begin{frame}[label=current]
    \frametitle{Model}
    \framesubtitle{}
    \begin{propo}
        Let $\mu_1(\bm{G})$ be the largest eigenvalue of $\bm{G}$,
        \footnote{$\mu_1(\bm{G})$ is well-define and larger than $0$ since all eigenvalues
        of a symmetric matrix $\bm{G}$ are real, and the diagnal of $\bm{G}$ is zero.}
        the matrix $\beta[\bm{I}-\lambda^*\bm{G}]$ is well-defined and nonnegative
        if and only if $\beta>\lambda\mu_1(\bm{G})$, thus the unique interior Nash equilibrium
        is given by $\bm{x}^*(\bm{\Sigma})=\frac{\alpha}{\beta+\gamma b(\mathscr{G},\lambda^*)}
        b(\mathscr{G},\lambda^*)$.
    \end{propo}
    \begin{itemize}
        \item As a decision-maker or policymaker, we may want to find the most
        influential player in the network to break or strengthen such effect.
        \item As a decision-maker or policymaker, we may want to find the most
           influential player in the network to break or strengthen such effect.
    \end{itemize}
\end{frame}
\subsection{Second line of dots} % new line of headline dots

\section{Section no. 2} 

\begin{frame}[allowframebreaks]
    \bibliography{ref}
\end{frame}

\end{document}
