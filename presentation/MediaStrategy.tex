%%%%%%%%%%%%%%%%%%%%%%%%%%%%%%%%%%%%%%%%%%%%%%%%%%%%%%%%%%%%%%%
\PassOptionsToPackage{subsection=false}{beamerouterthememiniframes} % delete subsection line
\documentclass{beamer}
\usepackage{tikz}
\usepackage{array}
\usepackage{url}
\usepackage{amsmath, amsthm, amsfonts, amssymb}
\usepackage{booktabs}
\usepackage{enumerate}
\usepackage[shortlabels]{enumitem}
\usepackage{pifont}
\usepackage{framed}
\usepackage{textcomp}
\usepackage{caption}
\usepackage{subcaption}
\usepackage{xeCJK}
\usepackage[natural, svgnames]{xcolor}
\usepackage{natbib}
\bibliographystyle{econ}
\usepackage{hyperref}
\hypersetup{
    colorlinks,
    citecolor=myblue,
    linkcolor=myblue,
    urlcolor=myred
}


%%%%%%%%%%%%%%%%%%%%%%%%%%%%%%%%%%%%%%%%%%%%%%%%%%%%%%%%%%%%%%%
% beamer theme setting
\usetheme{Szeged}
\usecolortheme{rose}
\usefonttheme{serif}

%%%%%%%%%%%%%%%%%%%%%%%%%%%%%%%%%%%%%%%%%%%%%%%%%%%%%%%%%%%%%%%
% unable the control buttons at bottom
\setbeamertemplate{navigation symbols}{}

% define my color
\definecolor{myred}{RGB}{192,46,46}
\definecolor{myblue}{RGB}{7,75,164}
\definecolor{mygreen}{rgb}{0.11,0.7,0.02}
\definecolor{myyellow}{rgb}{0.7,0.625,0.1}

% modify footer
\newcommand{\shortauthor}{\textcolor{myblue}{Yu-Chieh Kuo}}
\setbeamertemplate{footline}
{%
  \begin{beamercolorbox}[colsep=1.5pt]
      {upper separation line foot}
  \end{beamercolorbox}
  \hbox{%
      \begin{beamercolorbox}[wd=0.333333\paperwidth, ht=4ex, 
          dp=2.3ex, leftskip=.3cm plus1fill, rightskip=.3cm]
          {title in head/foot}%
          \usebeamerfont{title in head/foot}
          \parbox{.28\paperwidth}{\raggedright\inserttitle}
    \end{beamercolorbox}%
    \begin{beamercolorbox}[wd=0.333333\paperwidth, ht=4ex, 
        dp=2ex, center]{title in head/foot}%
      \usebeamerfont{author in head/foot}
      \insertframenumber/\inserttotalframenumber
    \end{beamercolorbox}%
    \begin{beamercolorbox}[wd=0.333333\paperwidth, ht=4ex,
        dp=2ex, leftskip=.3cm, rightskip=.3cm plus1fil]
        {title in head/foot}%
        \usebeamerfont{title in head/foot}
        \parbox{.28\paperwidth}{\raggedleft\shortauthor}
    \end{beamercolorbox}}
  \begin{beamercolorbox}[colsep=1.5pt]
      {lower separation line foot}
  \end{beamercolorbox}
}

%% modify theorem and block
%\usepackage{parskip}
%%\usepackage{microtype}
%\newlength{\prevparskip}
%
%\defbeamertemplate{block begin}{lines}{%
%    \par\vskip2ex plus 1fil%
%    \setlength{\prevparskip}{\parskip}%
%    \setlength{\parskip}{0pt}%
%    \begin{beamercolorbox}{block title}%
%        \usebeamerfont{block title}%
%        \vskip-0.6ex%
%        \insertblocktitle%
%        \vskip-2ex%
%        {\color{myblue}\rule{\textwidth}{1pt}}%
%    \end{beamercolorbox}%
%    \nointerlineskip%
%    \begin{beamercolorbox}[vmode]{block body}%
%        \usebeamerfont{block body}%
%        \vskip 1.2ex%
%}
%
%\defbeamertemplate{block end}{lines}{%
%    \end{beamercolorbox}%
%    \vskip-1ex%
%    \rule{\textwidth}{1pt}%
%    \setlength{\parskip}{\prevparskip}%
%    \vskip-1ex plus 1fil
%}
%
%\setbeamertemplate{blocks}[lines]

% modify section and subsection's entry
\setbeamertemplate{section in toc}{
  {\color{myblue}\inserttocsectionnumber.}~\inserttocsection}
\setbeamertemplate{subsection in toc}{
\hspace{1.2em}{\color{myblue}$\triangleright$}~\inserttocsubsection\par}

% modify itemize's entry and position
\setlist[itemize]{label=\textcolor{myblue}{$\triangleright$}}
\setlength{\leftmargini}{0pt}

% modify color
\setbeamercolor{block title}{fg=myblue, bg=white}
\setbeamercolor{block body}{fg=black, bg=white}
%\setbeamercolor{block title alerted}{use=structure, fg=white, bg=myred!90!white}
%\setbeamercolor{block body alerted}{use=structure, fg=black, bg=myred!20!white}
\setbeamercolor{frametitle}{fg=myblue}
\setbeamercolor{separation line}{bg=myblue}
\setbeamercolor{section in head/foot}{fg=myblue}
\setbeamercolor{subsectiopn in head/foot}{fg=myblue}
\setbeamercolor{section in toc}{fg=myblue}
\setbeamercolor{subsection in toc}{fg=myblue}
\setbeamercolor*{title}{fg=myblue}
\setlist[description]{format=\textcolor{myblue}}

% Modify margin setting
\justifying
\setbeamersize{text margin left=1.8em,text margin right=1.8em}
\setlength{\parindent}{2em}

\addtobeamertemplate{frametitle}{\setlength{\parindent}{0em}}{}
\addtobeamertemplate{block begin}{\setlength{\parindent}{0em}}{\setlength{\parindent}{2em}}
\addtobeamertemplate{block example begin}{\setlength{\parindent}{0em}}{\setlength{\parindent}{2em}}
\settowidth{\leftmargini}{\usebeamertemplate{itemize item}}
%\addtolength{\leftmargini}{\labelsep}

% modify font
\setCJKmainfont{思源宋體 TW}
\setmainfont{Philosopher}
\setbeamerfont{title}{size=\huge, series=\bfseries}
\newfontfamily\chancery{Apple Chancery}
\DeclareTextFontCommand{\textch}{\chancery}

%%%%%%%%%%%%%%%%%%%%%%%%%%%%%%%%%%%%%%%%%%%%%%%%%%%%%%%%%%%%%%%

\newcolumntype{R}{>{\displaystyle}r}
\newcolumntype{C}{>{\displaystyle}c}
\newcolumntype{L}{>{\displaystyle}l}

%%%%%%%%%%%%%%%%%%%%%%%%%%%%%%%%%%%%%%%%%%%%%%%%%%%%%%%%%%%%%%%
% title page information

\title{
    Media Platforms' Content Provision Strategies and Sources of Profits
}  
\author{Yu-Chieh Kuo\inst{1}} 
\date{\today} 

\institute[NTU]
{
    \inst{1}
    Department of Information Management,
    National Taiwan University
}

%%%%%%%%%%%%%%%%%%%%%%%%%%%%%%%%%%%%%%%%%%%%%%%%%%%%%%%%%%%%%%%

\AtBeginSection[]
{
    \begin{frame}[noframenumbering]
      \frametitle{}
      \tableofcontents[currentsection, hideallsubsections]
    \end{frame}
}

%%%%%%%%%%%%%%%%%%%%%%%%%%%%%%%%%%%%%%%%%%%%%%%%%%%%%%%%%%%%%%%

\newcommand{\hl}[1]{\textcolor{myblue}{#1}}
\newcommand{\rhl}[1]{\textcolor{myred}{#1}}
\newcommand{\adv}{\text{advertiser}}
\newcommand{\fc}{\textch{\hl{Free-content}}}
\newcommand{\nad}{\textch{\hl{No-ad}}}
\newcommand{\pcwa}{\textch{\hl{Paid-content-with-ads}}}
\newcommand{\dual}{\textch{\hl{Dual}}}
\newcommand{\free}{\textch{\hl{Free}}}
\newcommand{\pccm}{\text{Perfectly Competitive Content Market}}
\newcommand{\mcm}{\text{Monopoly Content Market}}
\newcommand{\mccm}{\text{Moderately Competitive Content Market}}
\newcommand{\eq}{\text{equilibrium}}
\newcommand{\Pp}[1]{\text{Proposition #1}}
\newcommand{\Int}{\text{Intuition}}

%%%%%%%%%%%%%%%%%%%%%%%%%%%%%%%%%%%%%%%%%%%%%%%%%%%%%%%%%%%%%%%


%\includeonlyframes{current}
\begin{document}

\begin{frame}
\titlepage
\end{frame}

\begin{frame}%[label=current]
    \frametitle{Background}
    \framesubtitle{}
    \begin{itemize}
        \item The media platforms grow and be one part
            of our life nowadays.
        \item We can observe that platforms adopt different
            strategies for monitization, for example,
            subscription, ad-only, dual-adoption.
        \item Moreover, UGC (User-Generated-Content) and PGC
            (Professionally-Generated-Content) platforms face
            different trade-off when determining their strategy.
        \item At the same time, users and advertisers have their
            desire for content and consumer.
            \begin{itemize}
                \item Users want to procure contents in their favor.
                \item Advertisers want to display their ads to their targets.
            \end{itemize}
        \item \hl{Platforms nowadays iteract with multi agents: users, \adv s,
            content suppliers.}
    \end{itemize}

\end{frame}


\begin{frame}
    \frametitle{Questions}
    \begin{itemize}
        \item How do platforms adopt their strategies
            under different scenarios with the interaction between content suppliers,
            consumers, and advertisers?
        \item What is an optimal strategy for a media platform under
            different scenario?
        \item How does consumers' desire for content and advertisers' desire
            for consumers affect a platform's content provision strategy?
        \item Specificly, we want to examine the optimal strategy \hl{under different
            content market structures}.
            \begin{itemize}
                \item We will consider \hl{perfect competition, monopoly,} and
                    \hl{moderate competition} in content market structures.
            \end{itemize}
    \end{itemize}
\end{frame}

\begin{frame}%[label=current]
    \frametitle{Literature Review}
    \framesubtitle{}
    \begin{itemize}
        \item Researchers highlight and analyze various issue for
            two-sided platforms in buyer-seller markets, such as platform competition
            in \citet{arms06}, pricing in \citet{weyl10}, and network asymmetry in
            \citet{ambrus09}.
        \item This research follows the related literature on two-sided media markets,
            and consider the empirical evidence of ads dislike from consumers in 
            \citet{wil08}, the desire of reaching consumer from \adv s in \citet{arg07}
            , and the condition to offer consumers an opinion of paying for no ads in
            \citet{shin19}.
        \item \hl{However, these works focus on the consumer and \adv\ side but abstract
            away from the content side of the market.} We extend to three-sided platforms
            and include the consumers' desire for content, then examine how platforms
            should allocate its limited bandwidth for content and ads.
    \end{itemize}
\end{frame}

\begin{frame}
    \frametitle{Outline}
    \tableofcontents
\end{frame} 

\section{Model Setting} 
\begin{frame}%[label=current]
    \frametitle{Consumer}
    \framesubtitle{Utility from procuring content}
    \begin{itemize}
        \item We start from a duopoly media markets where
            media pkatforms procure content from suppliers, 
            allocate a space for content and host the promotional
            messages of \adv s.
        \item The consumer's utility when joining platform $i$ to enjoy the content
            is $v\cdot\left(a_i-\frac{1}{2}a_i^2\right)$.
            \begin{itemize}
                \item $v$ represents how consumer desires the content.
                \item $a_i\in[0,1]$ is the proportion of platform $i$'s space
                    allocated for content; $1-a_i$ is the space for ads on
                    the other hand.
            \end{itemize}
        \item This formulation \hl{captures the reality that the utility of the
            incremental content is likely to be lower as the proportion of content
            increases.}
        \item Moreover, the marginal utility of content declines and may reach to a 
            balance of the potential revenue from consuemrs and ads.
    \end{itemize}
\end{frame}

\begin{frame}%[label=current]
    \frametitle{Consumer}
    \framesubtitle{Utility from joining platform}
    \begin{itemize}
        \item Consumer are heterogeneous in their preference for a platform,
            and we capture such the heterogeneity by \hl{Hotelling line}.
            \begin{itemize}
                \item Assume that consumers are uniformly distributed on $[0,1]$.
                \item A consumer located at a distance $x$ from platform $i$
                    experiences a disutility by $tx$, where $t$ captures the
                    consumers' sensitivity to platform characteristics.
            \end{itemize}
        \item The consumers pay a price $p_{iC}$ when joining platform $i$,
            and obtain the overall utility
            \[
                U_{iC}(x)=v\cdot\left(a_i-\frac{1}{2}a_i^2\right)
                -tx-p_{iC}.
            \]
    \end{itemize}
\end{frame}

\begin{frame}%[label=current]
    \frametitle{Advertisers}
    \framesubtitle{}
    \begin{itemize}
        \item Advertisers want to join a platform to promote their products
            and services to consumers.
        \item Advertisers' valuation of a consumer is $r_A$, and if they
            can reach $n_{iC}$ consumers through platform $i$, they obtain
            the utility $u_{iA}=r_A\cdot n_{iC}$.
        \item Assume that a platform can capture the entire surplus from 
            \adv s due to a scarce ad promotion space, the platform will
            set a promotion price at $p_{iA}=u_{iA}=r_A\cdot n_{iC}$ (\citet{shin15})
            and earn an ad revenue by $(1-a_i)p_{iA}$.
    \end{itemize}
\end{frame}

\begin{frame}%[label=current]
    \frametitle{Content Suppliers}
    \framesubtitle{}
    \begin{itemize}
        \item Platform purchase the content from the suppliers to serve their
            consumers.
        \item Let $c$ be the marginal cost of producing content and the supplier
            charges $p_S$ for a unit of content, their profit is
            $\Pi_S=(p_S-c)\cdot\sum_{i}a_i$. \\
            %\textcolor{myred}{(The author doesn't consider a differential pricing
            %for the content to simplify the model.)}
    \end{itemize}
\end{frame}

\begin{frame}%[label=current]
    \frametitle{Media Platforms}
    \framesubtitle{Profits and strategy space}
    \begin{itemize}
        \item Consequently, platform $i$ earns the profit
            $\Pi_{iP}=n_{iC}\cdot p_{iC}+(1-a_i)\cdot p_{iA}-a_i\cdot p_S$,
            and it faces an optimization problem by deciding
            \hl{the proportion allocated for content $a_i$, and
            the price charging to consumers $p_{iC}$}.
        \item A platform can adopt one of the following strategies.
    \end{itemize}
    \begin{description}
        \item[\fc:] The platform chooses to earn all its profits from \adv s
            by setting $0<a_i<1$ and $p_{iC}=0$.
        \item[\nad:] The platform chooses to earn all its profits from consumers
            by setting $a_i=1$ and $p_{iC}>0$.
        \item[\dual:] The platform earns profits from both consumers and \adv s
            and set $0<a_i<1$ and $p_{iC}>0$.
    \end{description}
\end{frame}

\begin{frame}%[label=current]
    \frametitle{Decision Sequence}
    \framesubtitle{}
    \begin{itemize}
        \item In the \hl{first stage}, \hl{the content supplier} sets a content price $p_S$.
        \item In the \textcolor{myred}{second stage}, 
            \textcolor{myred}{platforms} choose their strategies after observing
            $p_S$ and set $a_i$ and $p_{iC}$.
        \item In the \textcolor{myyellow}{third stage}, 
            \textcolor{myyellow}{consumers} determine which platform to join after 
            observing $a_i$ and $p_{iC}$.
    \end{itemize}
\end{frame}

\section{Perfectly Competitive Market}
\begin{frame}%[label=current]
    \frametitle{Consumer's Utility}
    \framesubtitle{}
    \begin{itemize}
        \item An infinite number of homogeneous content suppliers
            engage in the market and compete with each other.
        \item The \eq\ content supplier price is $p_S=c$ due to the 
            perfect competition.
        \item Assume that platform 1 is located at $x=0$ and platform 2
            at $x=1$, the consumers utility joining two platforms is
            \[
                \begin{array}{L}
                    U_{1C}(x)=v\cdot\left(a_1-\frac{1}{2}a_1^2\right)
                    -tx-p_{1C}\equiv V(a_1)-tx-p_{1C} \\
                    U_{2C}(x)=v\cdot\left(a_2-\frac{1}{2}a_2^2\right)
                    -t(1-x)-p_{2C}\equiv V(a_2)-t(1-x)-p_{2C},
                \end{array}
            \]
            and the indifferent consumer is located at
            \[
                x_0=\frac{1}{2}+\frac{V(a_1)-V(a_2)}{2t}-\frac{p_{1C}-p_{2C}}{2t}.
            \]
    \end{itemize}
\end{frame}

\begin{frame}%[label=current]
    \frametitle{Platform's strategy}
    \framesubtitle{}
    \begin{itemize}
        \item The mass of consumers joining platform 1 and 2 is
            $n_{1C}=x_0$ and $n_{2C}=1-x_0$, and platform $i$
            charges the ad promotion fee $p_{iA}=r_A\cdot n_{iC}$
            to \adv s.
    \end{itemize}
    \begin{description}
        \item[Lemma 1:] In this context, if the marginal cost of 
            producing content is too high ($c\geq\tau$), both platforms adopt
            \fc\ strategies or \dual\ strategies otherwise, where
            $\tau=\frac{1}{2}\left(-r_A+\frac{tv}{r_A}\right)$.
        \item[Intuition:] Since the content price $p_S=c$, platforms are
            not willing to buy too much content and display it 
            if $p_S$ is too high and will adopt a \fc\ strategy to 
            attract more users, and profit from \adv s. \nad\ will not be a case
            since \hl{we don't consider the user's disutility of watching ads here.}
            We will extend this disutility later.
%        \item[\textcolor{myred}{My comment:}] \textcolor{myred}{I don't feel
%            the assumption of excluding the user's disutility against ads
%            is feasible and closed to the reality.}
    \end{description}
\end{frame}

\begin{frame}%[label=current]
    \frametitle{Equilibrium and Profit}
    \framesubtitle{}
    \begin{itemize}
        \item If $c\geq\tau$, both platforms adopt \fc\ strategy
            then set $a_i^*=a^\free\equiv1-\frac{t(2c+r_A)}{\sqrt{tvr_A(2c+r_A)}}$
            and $p_{iC}^*=p_{iC}^\free\equiv0$, and obtain a profit
            at $\Pi_{iP}^\free=\frac{t(2c+r_A)^2}{2\sqrt{tvr_A(2c+r_A)}}-c$.
        \item If $c<\tau$, both platforms adopt \dual\ strategy
            then set $a_i^*=a^\dual\equiv\frac{-2c-r_A+v}{v}$
            and $p_{iC}^*=p_C^\dual\equiv\frac{-2cr_A-r_A^2+tv}{v}$,
            and obtain a profit at $\Pi_{iP}^\dual=\frac{4c^2-2(v-r_A)c+vt}{2v}$.
    \end{itemize}
\end{frame}

\begin{frame}%[label=current]
    \frametitle{Platform's strategy under different desire}
    \framesubtitle{}
    \begin{description}
        \item[\Pp{1(a)}:] A growing consumers' desire for content
            leads to the transition for platforms from using a \fc\
            strategy to \dual\ and earn less profits.
            \begin{itemize}
                \item Recall that in \hl{Lemma 1},
                    if the marginal cost of producing content is higher than
                    $\tau=\frac{1}{2}\left(-r_A+\frac{tv}{r_A}\right)$,
                    both platforms adopt \fc\ strategies.
                \item The value of $\tau$ increases as the consumers' desire
                    $v$ grows, which may change the relationship between
                    $c$ and $\tau$.
                \item However, due to $\frac{\partial {\Pi_{ip}^\free}}{\partial {v}}<0$
                    and $\frac{\partial {\Pi_{ip}^\dual}}{\partial {v}}<0$,
                    the profits decreases with the growing $v$.
                    Intuitively, the platform charges a higher
                    price due to a higher $v$, but also needs to offer more content
                    to attract customers, which ascends the cost and leaves
                    less space for ads.
            \end{itemize}
    \end{description}
\end{frame}

\begin{frame}%[label=current]
    \frametitle{Platform's strategy under different desire}
    \framesubtitle{}
    \begin{description}
        \item[\Pp{1(b)}:] A growing \adv s' desire for consumers ($r_A$)
            leads to the transition for platforms from \dual\ strategy
            to \fc\ and benefits platforms
            \begin{itemize}
                \item A growing $r_A$ decreases $\tau$ and thus
                    faciliates $c\geq\tau$.
                \item An increase in ads' desire for consumers motivates
                    platforms to increase the ads space by reducing the content's,
                    which enhances the ads profit but also decreases the
                    procurement cost for content.
            \end{itemize}
        \item[\Pp{2}:] When the price of content is sufficiently high,
            an increase in content price improves competing platforms' profit.
            \begin{itemize}
                \item Both platforms adopt a \fc\ strategy if 
                   the price of content is sufficiently high. An increase of 
                   the price of content motivates platforms to reduce the proportion
                   of content, which leads to a higher ads revenue and a lower
                   procurement cost.
            \end{itemize}
    \end{description}
\end{frame}

%\begin{frame}[label=current]
%    \frametitle{\pccm}
%    \framesubtitle{}
%    \begin{description}
%        \item[\Pp{1}:] A growing \adv s' desire for consumers ($r_A$) benefits
%            platforms.
%        \item[\Int:] Advertisers are willing to pay more if their desire
%            of promotion is stronger.
%        \item[\Pp{2}:] When the price of content is sufficiently high,
%            an increase in content price improves competing platforms' profit.
%        \item[\Int:] Remind that both platforms adopt a \fc\ strategy if 
%            the price of content is sufficiently high. An increase of 
%            the price of content motivates platforms to reduce the proportion
%            of content $a_i$, which leads to a higher ad revenue and a lower
%            procurement cost.
%    \end{description}
%\end{frame}

\begin{frame}%[label=current]
    \frametitle{Ads exposion}
    \framesubtitle{}
    \begin{itemize}
        \item Advertisers are more willing to pay more for promotion if
            a platform hosts more proportion of content since customers
            are expected to spend more time on platform and exposed to
            ads for a longer time.
        \item The price of ads is modified as $p_{iA}=a_i^k\cdot r_A\cdot n_{iC}$,
            where $0\leq k\leq1$ captures the sensitivity of the ad price to the
            propostion of content.
        \item The author claim that this modification consistently shows the platform
            has to face a balance for revenues between consumers and \adv s,
            and there exists a small $k\in(0,1]$ to make the result consistent with
            \Pp{2}.
    \end{itemize}
    \begin{description}
        \item[\textcolor{myred}{My comment:}] \textcolor{myred}{A larger proportion 
            of content excludes the exposion of ads. 
            Moreover, the author doesn't provide a proof for the claim.
            I think this modification should be improved.}
    \end{description}
\end{frame}

\section{Monopoly Market}
\begin{frame}%[label=current]
    \frametitle{Content Supplier's Decision}
    \framesubtitle{}
    \begin{itemize}
        \item The content supplier has no market power in the previous setting.
            Now we examine the case that the content supplier can set the content price
            $p_S$ and then discuss the interaction between the content supplier and 
            the platform.
        \item The content supplier determines the content price $p_S$ 
            to maximize its profit by anticipating
            the platform's strategy, that is,
            \[
                \max\ \Pi_S=(p_S-c)\sum a_i(p_S).
            \]
    \end{itemize}
\end{frame}

\begin{frame}%[label=current]
    \frametitle{Platform's Strategy and Profit}
    \framesubtitle{}
    \begin{itemize}
        \item Platforms' strategies are consistent with \hl{Proposition 1 and 2}.
            We want to observe the interactions between platforms and content suppliers
            for the different desire from consumers and advertisers.
    \end{itemize}
    \begin{description}
        \item[\Pp{3(a)}:] The stronger consumers' desire for content
            benefits the content supplier regardless of platforms' strategies.
            \begin{itemize}
                \item Such desire motivates platforms to purchase more 
                    content in all strategies and promotes the content supplier's
                    revenue.
            \end{itemize}
        \item[\Pp{3(b)}:] The stronger \adv s' desire for consumer
            hurts the content supplier's revenue under a \dual\ strategy
            but benefits it under a \fc\ strategy.
            \begin{itemize}
                \item Such desire motivates platforms to increase the space of ads
                    and consequently demands less content. \textcolor{myred}{However,
                    both I and the author cannot give the latter an intuition, and I have no
                    choice but to treat it as a mathematical result.}
            \end{itemize}
    \end{description}
\end{frame}

\begin{frame}%[label=current]
    \frametitle{Content Price}
    \framesubtitle{}
    \begin{itemize}
        \item Given different content cost, platforms select different strategies.
        \item Content cost alters to $p_S$ in this context, but the strategy selecting criteria
            for platforms is consistent in the previous one, that is, depending on 
            the cost of content.
        \item If $p_S<\tau$, platforms choose \dual\ strategy and the 
            demand for content in this case is
            $a^\dual(p_S)=\frac{-2p_S-r_A+v}{v}$; platforms choose \fc\
            and have the demand $a^\free(p_S)=1-\frac{t(2p_S+r_A)}{\sqrt{tvr_A(2p_S+r_A)}}$.
    \end{itemize}
    \begin{description}
        \item[\Pp{4}:] A monopoly content supplier may not extract all surplus
            from the platforms.
            \begin{itemize}
                \item Since platforms are able to profit from ads, \hl{they can
                    balance the profit from both channels carefully,
                    which is different with the traditional market.}
            \end{itemize}
    \end{description}
\end{frame}

\section{Moderately Competitive Market}
\begin{frame}[label=current]
    \frametitle{Moderate Competition on Content Price}
    \framesubtitle{}
    \begin{itemize}
        \item In addition to the previous two polar cases, we observe
            only a few content suppliers and moderate competition
            in the content market in some markets.
        \item Consider a duopoly content market. Let $p_{jS}$ be the price of 
            content from supplier $j$, and $h_i$ be the proportion of 
            content platform $i$ buys from supplier 1 and $1-h_i$
            from supplier 2.
        \item Given the content procurement decision, the profits of platform $i$ is
            \[
                \Pi_{iP}=n_{iC}p_{iC}+(1-a_i)p_{iA}
                -a_i\cdot h_i\cdot p_{1S}
                -a_i\cdot(1-h_i)\cdot p_{2S}.
            \]
        \item The corresponding profits of the two content suppliers are
            \[
                \begin{array}{RCL}
                    \Pi_{1S}=p_{1S}\cdot\sum_{i}(a_i\cdot h_i)
                    & \text{and} &
                    \Pi_{2S}=p_{2S}\cdot\sum_{i}\left(a_i\cdot(1-h_i)\right).
                \end{array}
            \]
    \end{itemize}
\end{frame}

\begin{frame}[label=current]
    \frametitle{Competition among Content Suppliers}
    \framesubtitle{}
    \begin{itemize}
        \item We assume two content suppliers are horizontally different to the platforms.
            Their difference factors, such as integration of systems, may influence platforms'
            decision but \hl{not affect the utility for consumers}.
            \begin{itemize}
                \item This allows us to focus on how mere competition between content
                    suppliers affects content price.
            \end{itemize}
        \item We use the marginal substitution rate (MRS) of the suppliers as a metric, and
            MRS gives the change in the demand when the content price increases by 1\%,
            where $MRS_i(p_{1S},p_{2S})\equiv\left|\frac{\partial h/h}{\partial p_{iS}/p_{iS}}\right|$,
            and $h\equiv h_1=h_2$.
    \end{itemize}
\end{frame}

\begin{frame}[label=current]
    \frametitle{Moderate Competition in Content Price}
    \framesubtitle{}
    \begin{itemize}
        \item A high MRS suggests that the content suppliers are more substitutable
            and the market is more competitive and vice versa.
        \item In a traditional one-sided market, if the seller increases the price,
            the buyer switches to the opposite firm.
        \item \hl{However, in a multi-sided media market, if one seller (content supplier)
            increases the price, the buyer (platforms) has two options: decrease the
            proportion of content, or switch to the competing supplier}
            \begin{itemize}
                \item \hl{This reduces the competition for the content suppliers in price.}
                    Content suppliers don't extremely bear the price competition compared with that
                    in a monopoly market, and the equilibrium price in a duopoly market is
                    higher than a monopoly one.
            \end{itemize}
    \end{itemize}
\end{frame}

\begin{frame}[label=current]
    \frametitle{Ads Disutility for Consumers}
    \framesubtitle{}
    \begin{itemize}
        \item In the original model, we don't consider the nuisance of ads for consumers.
            Here we extend the model by including a direct disutility of ads.
        \item Let $d_C\geq0$ be the consumers' dislike for a unit space of ads, then the
            utility of a consumer at distance $x$ from platform $i$ is 
            \[
                U_{iC}(x)=v\left(a_i-\frac{1}{2}a_i^2\right)\hl{-d_C\cdot(1-a_i)}
                -tx-p_{iC}.
            \]
            \begin{itemize}
                \item The original model can be regarded as a special case for $d_C=0$.
            \end{itemize}
    \end{itemize}
\end{frame}

\begin{frame}[label=current]
    \frametitle{Platform's Strategy When Ads Disutility Matters}
    \framesubtitle{}
    \begin{description}
        \item[\Pp{5}:] Even if consumers derive disutility from seeing ads,
            platforms never adopt a \nad\ strategy in a monopoly content supplier makert.
            However, in a competitive market, \nad\ is possible to adopted.
            \begin{itemize}
                \item Recall that a platform chooses \nad\ strategy only when
                    the content price is sufficiently low.
                \item However, \nad\ strategy leads platforms to buying the maximum amount
                    of content, which encourage the monopoly suppliers to raise the price.
                \item Consequently, platforms cater to \adv s and thus adopt \fc\ or
                    \dual\ strategy rather than \nad.
                \item In the competitive market, the content price is possibly low,
                    at meanwhile the disutility from ads restricts the benefits of ads.
            \end{itemize}
    \end{description}
\end{frame}

\begin{frame}[label=current]
    \frametitle{Practical Examples}
    \framesubtitle{}
    \begin{itemize}
        \item Media platforms such as Netflix and Spotify, who procure
            content from competing content suppliers, usually offer the content to consumers
            without ads.
        \item In contrast, TV stations broadcasting the Olympics of FIFA World Cup,
            which providecd by suppliers with some marketing power,
            typically host ads and also charge viewers.
    \end{itemize}
\end{frame}

\section{Conclusion}
\begin{frame}[label=current]
    \frametitle{Managerial Significance}
    \framesubtitle{}
    \begin{itemize}
        \item It is difficult for a manager to conjecture
            how the different sides of a media market might interact
            in a given situation, however, this paper provides useful
        \item Further research may put the quality of content, platform in-house
            content supplier, and any possible extension into consideration.
    \end{itemize}
\end{frame}

\begin{frame}[allowframebreaks]
    \bibliography{ref}
\end{frame}

%\end{CJK*}
\end{document}

