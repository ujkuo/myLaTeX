%%%%%%%%%%%%%%%%%%%%%%%%%%%%%%%%%%%%%%%%%%%%%%%%%%%%%%%%%%%%%%%
\PassOptionsToPackage{subsection=false}{beamerouterthememiniframes} % delete subsection line
\documentclass{beamer}
\usepackage{tikz}
\usepackage{array}
\usepackage{url}
\usepackage{amsmath, amsthm, amsfonts, amssymb, bm}
\usepackage{booktabs}
\usepackage{enumerate}
\usepackage[shortlabels]{enumitem}
%\usepackage{pifont}
\usepackage{framed}
\usepackage{textcomp}
\usepackage{caption}
\usepackage{dsfont}
\usepackage{subcaption}
\usepackage{xeCJK}
\usepackage[natural, svgnames]{xcolor}
\usepackage{natbib}
\bibliographystyle{econ}
\usepackage{hyperref}
\hypersetup{
    colorlinks,
    citecolor=myblue,
    linkcolor=myblue,
    urlcolor=myred
}

\usepackage[mathscr]{eucal}
\usepackage{tcolorbox}
%\everymath{\displaystyle}

%%%%%%%%%%%%%%%%%%%%%%%%%%%%%%%%%%%%%%%%%%%%%%%%%%%%%%%%%%%%%%%
% beamer theme setting
\usetheme{Szeged}
\usecolortheme{rose}
\usefonttheme{serif}

%%%%%%%%%%%%%%%%%%%%%%%%%%%%%%%%%%%%%%%%%%%%%%%%%%%%%%%%%%%%%%%
% unable the control buttons at bottom
\setbeamertemplate{navigation symbols}{}

% define my color
\definecolor{myred}{RGB}{192,46,46}
\definecolor{myblue}{RGB}{7,75,164}
\definecolor{mygreen}{rgb}{0.11,0.7,0.02}

% modify footer
\newcommand{\shortauthor}{\textcolor{myblue}{Yu-Chieh Kuo}}
\setbeamertemplate{footline}
{%
  \begin{beamercolorbox}[colsep=1.5pt]
      {upper separation line foot}
  \end{beamercolorbox}
  \hbox{%
      \begin{beamercolorbox}[wd=0.333333\paperwidth, ht=4ex, 
          dp=2.3ex, leftskip=.3cm plus1fill, rightskip=.3cm]
          {title in head/foot}%
          \usebeamerfont{title in head/foot}
          \parbox{.28\paperwidth}{\raggedright\inserttitle}
    \end{beamercolorbox}%
    \begin{beamercolorbox}[wd=0.333333\paperwidth, ht=4ex, 
        dp=2ex, center]{title in head/foot}%
      \usebeamerfont{author in head/foot}
      \insertframenumber/\inserttotalframenumber
    \end{beamercolorbox}%
    \begin{beamercolorbox}[wd=0.333333\paperwidth, ht=4ex,
        dp=2ex, leftskip=.3cm, rightskip=.3cm plus1fil]
        {title in head/foot}%
        \usebeamerfont{title in head/foot}
        \parbox{.28\paperwidth}{\raggedleft\shortauthor}
    \end{beamercolorbox}}
  \begin{beamercolorbox}[colsep=1.5pt]
      {lower separation line foot}
  \end{beamercolorbox}
}
%% modify theorem and block
%\usepackage{parskip}
%%\usepackage{microtype}
%\newlength{\prevparskip}
%
%\defbeamertemplate{block begin}{lines}{%
%    %\par\vskip2ex plus 1fil%
%    \setlength{\prevparskip}{\parskip}%
%    \setlength{\parskip}{0pt}%
%    \begin{beamercolorbox}{block title}%
%        \usebeamerfont{block title}%
%        \vskip-0.6ex%
%        \insertblocktitle%
%        \vskip-2ex%
%        {\color{myblue}\rule{\textwidth}{1pt}}%
%    \end{beamercolorbox}%
%    \nointerlineskip%
%    \begin{beamercolorbox}[vmode]{block body}%
%        \vskip-1ex%
%        \usebeamerfont{block body}%
%        \vskip 1.2ex%
%}
%
%\defbeamertemplate{block end}{lines}{%
%    \end{beamercolorbox}%
%    \vskip-1ex%
%    \rule{\textwidth}{1pt}%
%    \setlength{\parskip}{\prevparskip}%
%    \vskip-1ex plus 1fil
%}
%
%\setbeamertemplate{blocks}[lines]



% modify section and subsection's entry
\setbeamertemplate{section in toc}{
  {\color{myblue}\inserttocsectionnumber.}~\inserttocsection}
\setbeamertemplate{subsection in toc}{
\hspace{1.2em}{\color{myblue}$\triangleright$}~\inserttocsubsection\par}

% modify itemize's entry and position
\setlist[itemize]{label=\textcolor{myblue}{$\triangleright$}}
\setlength{\leftmargini}{0pt}

% modify color
\setbeamercolor{block title}{fg=myblue, bg=white}
\setbeamercolor{block body}{fg=black, bg=white}
%\setbeamercolor{block title alerted}{use=structure, fg=white, bg=myred!90!white}
%\setbeamercolor{block body alerted}{use=structure, fg=black, bg=myred!20!white}
\setbeamercolor{frametitle}{fg=myblue}
\setbeamercolor{separation line}{bg=myblue}
\setbeamercolor{section in head/foot}{fg=myblue}
\setbeamercolor{subsectiopn in head/foot}{fg=myblue}
\setbeamercolor{section in toc}{fg=myblue}
\setbeamercolor{subsection in toc}{fg=myblue}
\setbeamercolor*{title}{fg=myblue}
\setlist[description]{format=\textcolor{myblue}}

% Modify margin setting
\justifying
\setbeamersize{text margin left=1.8em,text margin right=1.8em}
\setlength{\parindent}{2em}

\addtobeamertemplate{frametitle}{\setlength{\parindent}{0em}}{}
\addtobeamertemplate{block begin}{\setlength{\parindent}{0em}}{\setlength{\parindent}{2em}}
\addtobeamertemplate{block example begin}{\setlength{\parindent}{0em}}{\setlength{\parindent}{2em}}
\settowidth{\leftmargini}{\usebeamertemplate{itemize item}}
%\addtolength{\leftmargini}{\labelsep}

% modify font
\setCJKmainfont{思源宋體 TW}
\setmainfont{Philosopher}
%\setmainfont{Apple Chancery}
%\setmainfont{Lato}
\setbeamerfont{title}{size=\huge, series=\bfseries}
\newfontfamily\chancery{Apple Chancery}
\DeclareTextFontCommand{\textch}{\chancery}

%%%%%%%%%%%%%%%%%%%%%%%%%%%%%%%%%%%%%%%%%%%%%%%%%%%%%%%%%%%%%%%

\newcolumntype{R}{>{\displaystyle}r}
\newcolumntype{C}{>{\displaystyle}c}
\newcolumntype{L}{>{\displaystyle}l}

%%%%%%%%%%%%%%%%%%%%%%%%%%%%%%%%%%%%%%%%%%%%%%%%%%%%%%%%%%%%%%%
% title page information

\title{
    Who's Who in Networks. WANTED: The Key Player
}  
\author{Yu-Chihe Kuo\inst{1}} 
\date{\today} 

\institute[NTU]
{
    \inst{1}
    Department of Information Management,
    National Taiwan University
}

%%%%%%%%%%%%%%%%%%%%%%%%%%%%%%%%%%%%%%%%%%%%%%%%%%%%%%%%%%%%%%%

\AtBeginSection[]
{
    \begin{frame}[noframenumbering]
      \frametitle{}
      \tableofcontents[currentsection, hideallsubsections]
    \end{frame}
}

%%%%%%%%%%%%%%%%%%%%%%%%%%%%%%%%%%%%%%%%%%%%%%%%%%%%%%%%%%%%%%%

\newcommand{\bcm}{\text{Bonacich centrality measure }}
\newcommand{\hl}[1]{\textcolor{myblue}{#1}}
\newcommand{\lb}[1]{\underline{#1}}
\newcommand{\sij}{\sigma_{ij}}
\newcommand{\gij}{g_{ij}}
\newcommand{\lbs}{\lb{\sigma}}
\newcommand{\ubs}{\bar{\sigma}}
\newcommand{\bb}[1]{\mathbb{#1}}
\newcommand{\eu}[1]{\mathscr{#1}}
\newcommand{\ds}[1]{\mathds{#1}}
\newcommand{\eug}{\mathscr{G}}
\newcommand{\Ne}{\text{Nash equilibrium }}
\newcommand{\thm}{\textch{Theorem }}

%%%%%%%%%%%%%%%%%%%%%%%%%%%%%%%%%%%%%%%%%%%%%%%%%%%%%%%%%%%%%%%

%\includeonlyframes{current}
\begin{document}

\begin{frame}%[label=current]
\titlepage
\end{frame} 

\begin{frame}%[label=current]
    \frametitle{Motivation}
    \framesubtitle{}
    \begin{itemize}
        \item A network consists of several individuals linking to each other or not, and there
            may be some groups in a network.
        \item The dependence of individual outcomes on group behavior is often referred to as
            \hl{peer effects}.
            \begin{itemize}
                \item In standard peer effects models, this dependence is 
                    homogeneous across memebrs and corresponds to an \hl{average} group influence.
                \item As a decision-maker or policymaker, we may want to find the most
                    influential player in the network to break or strengthen such effect.
            \end{itemize}
        \item \hl{What if this intergroup externality is heterogeneous cross group members and 
            varies accross individuals with their level of group exposure?}
    \end{itemize}
\end{frame}

\begin{frame}%[label=current]
    \frametitle{Literature Reviews}
    \framesubtitle{}
    \begin{itemize}
        \item The first related measure was proposed by \citet{bonacich}, and some 
            sociologists establish the network analysis \citet{social} as well.
        \item However, the \bcm fails to internalize all the network payoff externalities
            agents exert on each other, whereas the intercentrality measure internalizes them all. 
        \item This research extended the \bcm and propose a new centrality measure
            \hl{based on the planner's optimality (collective) perspectives} instead of strategic
            (individual) considerations.
    \end{itemize}
\end{frame}

\begin{frame}
    \frametitle{Outline}
    \tableofcontents[hideallsubsections]
\end{frame} 

\section{Model Setting} 
\begin{frame}%[label=current]
    \frametitle{Utility and the Game}
    \framesubtitle{}
    \begin{itemize}
        \item Each player $i=1,\cdots,n$ selects an effort $x_i\geq0$ and obtains the bilinear
            utility $u_i(x_1,\cdots,x_n)=\alpha_ix_i+\frac{1}{2}\sigma_{ii}x_i^2+
            \sum_{j\neq i}\sigma_{ij}x_ix_j$, which is strictly concave in own effort, and the
            utility is linear-quadratic.
        \item Bilateral influences are captured by the cross-derivatives 
            $\frac{\partial^2 {u_i}}{\partial {x_i}\partial {x_j}}=\sigma_{ij}$ and can be of either
            sign. 
            \begin{itemize}
                \item For example, if $\sigma_{ij}>0$, an increase in $j$'s efforts triggers
                    a upwards shift \hl{in $i$'s response}, and we say $i$ and $j$'s efforts
                    are \hl{strategic complements from $i$'s perspective}.
            \end{itemize}
        \item Simplifying, we set $\alpha_i=\alpha>0$, $\sigma_{ii}=\sigma$, and 
            denote by $\bm{\Sigma}\equiv[\sigma_{ij}]$ the square matrix of cross-effects.
        \item Moreover, we define $\lbs\equiv\min\{\sigma_{ij}|i\neq j\}$ and 
            $\ubs\equiv\max\{\sigma_{ij}|i\neq j\}$ and assume that
            $\sigma<\min\{\lbs,0\}$.
    \end{itemize}
\end{frame}

\begin{frame}%[label=current]
    \frametitle{Cross-effects}
    \framesubtitle{}
    \begin{itemize}
        \item The next step is to discuss how to capture the relative complementarity in efforts
            between $(i,j)$.
            \begin{itemize}
                \item There are some discussion based on the sign of $\lbs$, and we skip it and use the
                    result directly.
            \end{itemize}
        \item Define $\gamma\equiv-\min\{\lbs,0\}\geq0$ and $\lambda\equiv\ubs+\gamma\geq0$.
            \footnote{In fact, $\lambda=0$ has Lebesgue measure zero.} and let
            $\gij\equiv\frac{\sij+\gamma}{\lambda}$ for $i\neq j$ and $g_{ii}=0$.
            \footnote{The result is robust in the case $g_{ii}=1$. This case is less
            economic intuitive said by the author.} Therefore,
            \hl{$0\leq \gij\leq1$ is a parameter measuring the relationship in efforts within $(i,j)$
            from $i$'s perspective, and the matrix $\bm{G}=[\gij]$ interprets the adjacency
            matrix of the network.}
    \end{itemize}
\end{frame}

\begin{frame}%[label=current]
    \frametitle{Bilateral Influences}
    \framesubtitle{}
    \begin{itemize}
        \item Let $\sigma=-\beta-\gamma$ for $\beta>0$ 
            satisfying the assumption of $\sigma<\min\{\lbs,0\}$ WLOG, and
            denote by $\bm{I}$ the identity matrix and $\bm{U}$ the matrix of ones, 
            where both are $n\times n$ matrices, we can decompose the matrix $\bm{\Sigma}$ as
            \hl{$\bm{\Sigma}=-\beta\bm{I}-\gamma\bm{U}+\lambda\bm{G}$}.
            \begin{itemize}
                \item Therefore, bilateral influences result from
                    the combination of \hl{an individual effect by $-\beta\bm{I}$,
                    the global interaction effect by $-\gamma\bm{U}$, and the local
                    interaction effect by $\lambda\bm{G}$}.
            \end{itemize}
        \item We can rewrite the utility function following the decomposition of $\bm{\Sigma}$ as
            $u_i(x_1,\cdots,x_n)=\alpha x_i-\frac{1}{2}(\beta-\gamma)x_i^2
            -\gamma\sum_{j=1}^nx_ix_j+\lambda\sum_{j=1}^n\gij x_ix_j$ for all $i=1,\cdots,n$.
    \end{itemize}
\end{frame}

\begin{frame}%[label=current]
    \frametitle{The \bcm}
    \framesubtitle{}
    \begin{itemize}
        \item Before moving to the equilibrium analysis, we define a network centrality measure
            extended by \bcm for the further use.
        \item Remind that the matrix $\bm{G}^k$ tracks the indirect connections in the network: 
            $\gij^k$ measures the number of paths of length $k\geq1$ in the network $\mathscr{G}$ from
            $i$ to $j$.
        \item Given a sufficiently small scalar $a\geq0$, we define the matrix
            $\bm{M}(\eu{G},a)=[\bm{I}-a\bm{G}]^{-1}=\sum_{k=0}^{+\infty}a^k\bm{G}^k$.
            $a$ represents a decay factor to scale down the weight of long paths.
        \item The vector of Bonacich centrality in $\eu{G}$ is
            $b(\eu{G},a)=[\bm{I}-a\bm{G}]^{-1}\cdot\ds{1}$, and the Bonacich centrality of node $i$
            is $b_i(\eu{G},a)=\sum_{j=1}^nm_{ij}(\eu{G},a)$.
    \end{itemize}
\end{frame}

\begin{frame}%[label=current]
    \frametitle{The \bcm}
    \framesubtitle{}
    \begin{itemize}
        \item We can separate the Bonacich centrality into two parts: from $i$ to $i$ itself
            and of all the outer path from $i$ to every other $j\neq i$. That is,
            $b_i(\eu{G},a)=\sum_{j=1}^nm_{ij}(\eu{G},a)=m_{ii}(\eug,a)+\sum_{j\neq i}m_{ij}(\eug,a)$.
            \begin{itemize}
                \item $m_{ii}(\eug,a)\geq1$ by definition and thus $b_i(\eug,a)\geq1$.
            \end{itemize}
    \end{itemize}
\end{frame}

\section{Equilibrium Analysis}
\begin{frame}%[label=current]
    \frametitle{Nash Equilibrium}
    \framesubtitle{}
    \begin{itemize}
        \item Recall that the utility function can be describe as 
            $u_i(\bm{x})=\alpha_ix_i+\frac{1}{2}\bm{\Sigma x}^2$. A Nash equilibrium in pure strategies 
            $\bm{x}^*\in\mathbb{R}^n_{+}$ is to solve 
            $\frac{\partial {u_i}(\bm{x}^*)}{\partial {x_i}}=0$ and $x_i^*>0$,
            that is, $-\bm{\Sigma}\cdot\bm{x}^*=[\beta\bm{I}+\gamma\bm{U}-\lambda\bm{G}]\cdot\bm{x}^*
            =\alpha\cdot\ds{1}$.
        %\vskip-2ex%
        \item Using the fact that $\bm{U}\cdot\bm{x}^*=x^*\cdot\ds{1}$ and define 
            $\lambda^*\equiv\frac{\lambda}{\beta}$, the FOC reduces to
            $\beta[\bm{I}-\lambda^*\bm{G}]\cdot\bm{x}^*=(\alpha-\gamma x^*)\cdot\ds{1}$.
    \end{itemize}
%    \begin{theorem}
%        Let $\mu_1(\bm{G})$ be the largest eigenvalue of $\bm{G}$,
%            \footnote{$\mu_1(\bm{G})$ is well-define and larger than $0$ since all eigenvalues
%            of a symmetric matrix $\bm{G}$ are real, and the diagnal of $\bm{G}$ is zero.}
%            the matrix $\beta[\bm{I}-\lambda^*\bm{G}]$ is well-defined and nonnegative
%            if and only if $\beta>\lambda\mu_1(\bm{G})$, thus the unique interior Nash equilibrium
%            is given by $\bm{x}^*(\bm{\Sigma})=\frac{\alpha}{\beta+\gamma b(\eug,\lambda^*)}
%            b(\eug,\lambda^*)$.
%    \end{theorem}
    \begin{description}
        \item[Theorem 1:] Let $\mu_1(\bm{G})$ be the largest eigenvalue of $\bm{G}$,
            \footnote{$\mu_1(\bm{G})$ is well-define and larger than $0$ since all eigenvalues
            of a symmetric matrix $\bm{G}$ are real, and the diagnal of $\bm{G}$ is zero.}
            the matrix $\beta[\bm{I}-\lambda^*\bm{G}]$ is well-defined and nonnegative
            if and only if $\beta>\lambda\mu_1(\bm{G})$, thus the unique interior Nash equilibrium
            is given by $\bm{x}^*(\bm{\Sigma})=\frac{\alpha}{\beta+\gamma b(\eug,\lambda^*)}
            b(\eug,\lambda^*)$.
    \end{description}
\end{frame}

\begin{frame}%[label=current]
    \frametitle{Parameters Analysis}
    \framesubtitle{}
    \begin{itemize}
        \item Given the unique Nash equilibrium 
            $\bm{x}^*(\bm{\Sigma})=\frac{\alpha}{\beta+\gamma b(\eug,\lambda^*)}
            b(\eug,\lambda^*)$, we want to analyze how three different effects influence
            the equilibrium.
            \begin{itemize}
                \item If the matrix of cross-effects $\bm{\Sigma}$ reduces to $\lambda\bm{G}$, that is,
                    $\beta=\gamma=0$, there exists no Nash equilibrium.
                \item If $\bm{\Sigma}$ reduces to $-\beta\bm{I}-\gamma\bm{U}$, that is, $\lambda=0$,
                    the \Ne is unique.
            \end{itemize}
        \item The existence and uniqueness of \Ne are proven by \citet{debreu}. We emphasize the 
            economic meaning.
            \begin{description}
                \item[\textcolor{myred}{My explanation:}] 
                    If the cross-effects will not be affected by your effort and
                    the substitutability in efforts across all pairs of players
                    , you may prefer doing nothing and result in an effort $x_i=0$
                    to obtain a higher utility,
                    which contradicts the condition of an interior \Ne.
            \end{description}
    \end{itemize}
\end{frame}

\begin{frame}%[label=current]
    \frametitle{Individual's Contribution to the Aggregate Equilibrium}
    \framesubtitle{}
    \begin{itemize}
        \item The Bonacich-Nash equilibrium expression also implies that each individual 
            contributes to the aggregate equilibrium outcome in proportion to their network centrality:
            $x_i^*(\bm{\Sigma})=\frac{b_i(\eug,\lambda^*)}{b(\eug,\lambda^*)}x^*(\bm{\Sigma})$.
        \item This indicates that the intergroup externality \hl{is not an average influence
            but a weighted one heterogeneous across members.}
            \begin{description}
                \item[\textcolor{myred}{My explanation:}] 
                    An unbalanced influence across memebrs allows us to find the most
                    significant player.
            \end{description}
    \end{itemize}
\end{frame}

\section{Find the Key Player}
\begin{frame}%[label=current]
    \frametitle{Identification Criterion}
    \framesubtitle{}
    \begin{itemize}
        \item After solving the \Ne and related issues, we go back to the main topic:
            how to find the key player in a network.
        \item The idea is: we want to reduce the player optimally to \hl{maximize
            the difference between the value of aggregate Nash equilibrium from this removal.} 
            Formally, we 
            solve an optimization problem \hl{$\max\{\bm{x}^*(\bm{\Sigma})-\bm{x}^*(\bm{\Sigma}_{-i})\}$.}
            \begin{itemize}
                \item This is equivalent to solve $\min\{\bm{x}^*(\bm{\Sigma}_{-i})|i=1,\cdots,n\}$.
            \end{itemize}
        \item Let $i^*$ be a solution to the optimization problem. We call $i^*$ the key player,
            which means removing $i^*$ from the initial network has the \hl{largest overall impact
            on the aggregate equilibrium level.}
    \end{itemize}
\end{frame}

\begin{frame}%[label=current]
    \frametitle{New Measure: Intercentrality}
    \framesubtitle{}
    \begin{itemize}
        \item Remind that the \bcm only counts the number of paths stemming from player $i$,
            which doesn't include \hl{the contributions of player $i$ toward other player $j\neq i$.}
        \item Therefore, the author proposed the \hl{intercentrality
            $c_i(\eug,a)=\frac{b_i(\eug,a)^2}{m_{ii}(\eug,a)}$,}
            to capture such combined centrality.
                    %\small
                    \setlength{\arraycolsep}{2.5pt}
                    %\medmuskip = 1mu
                    \[
                        \raggedleft
                        \begin{array}{LL}
                            \text{\textcolor{myblue}{$\triangleright$}}\,\,\,\,
                            c_i(\eug,a)&=\frac{b_i(\eug,a)^2}{m_{ii}(\eug,a)}
                            =\frac{\left(\sum_{j=1}^nm_{ij}(\eug,a)\right)^2}{m_{ii}(\eug,a)}\\[3mm]
                            &=\frac{\left(m_{ii}(\eug,a)+\sum_{j\neq i}m_{ij}(\eug,a)\right)^2}
                            {m_{ii}(\eug,a)}\\[3mm]
                            &=
                            b_i(\eug,a)+
                            \frac{\sum_{j\neq i}m_{ij}(\eug,a)\cdot b_i(\eug,a)}{m_{ii}(\eug,a)}.
                        \end{array}
                    \]
    \end{itemize}
\end{frame}

\begin{frame}%[label=current]
    \frametitle{Intercentrality and the Key Player}
    \framesubtitle{}
    \begin{itemize}
        \item In fact, removing a player from a network has two effects:
            \begin{itemize}
                \item Fewer players contribute to the aggregate activity level 
                    (direct effect).
                \item The network topology is modified, which forces the remaining players
                    to adopt different actions (indirect effect).
            \end{itemize}
        \item Therefore, we want to catch the key play by using the intercentrality.
    \end{itemize}
    \begin{description}
        \item[Theorem 2:] The key player $i^*$ who solves the optimization problem
            $\min\{x^*(\bm{\Sigma}_{-i})|i=1,\cdots,n\}$ \hl{has the highest intercentrality}
            of parameter $\lambda^*$ in $\eug$, that is, 
            $c_{i^*}(\eug,\lambda^*)\geq c_{-i^*}(\eug,\lambda^*)$.
    \end{description}
\end{frame}

\begin{frame}%[label=current]
    \frametitle{Example}
    \framesubtitle{}
    \begin{itemize}
        \item For example, consider the following network $\eug$. Player 1 bridges together
            two groups, and removing player 1 disrupts the network.
        \item However, removing player 2 decreases maximally the total number of network links.
    \end{itemize}
    \includegraphics[width=\textwidth]{exp}
\end{frame}

\begin{frame}%[label=current]
    \frametitle{Example}
    \framesubtitle{}
    \begin{itemize}
        \item The computational result shows that as the value of $a$ (the decay factor of long paths)
            is low, player 2 has the highest Bonacich centrality and also is the key player;
            however, when $a$ is high, player 2 is not the key player but player 1 is.
        \item By considering indirect effects, removing player 1 has the highest joint direct
            and indirect effect on \hl{aggregate outcome}.
    \end{itemize}
    \centering
    \includegraphics[width=0.8\textwidth]{table}
\end{frame}

\section{Discussion}
\begin{frame}[label=current]
    \frametitle{Utility Form}
    \framesubtitle{}
    \begin{itemize}
        \item There is a number of possible extension of the work.
        \item The first is that the analysis is restricted to linear-quadratic utility that
            capture linear externality in player's actions.
            \begin{itemize}
                \item They use FOC to find the interior equilibrium and leads to the Bonacich-Nash
                    linkage.
            \end{itemize}
        \item Linear-quadratic utilities are commonly used in economic models.
        \item It can be extended to more general cases, such as non-linear externalities.
    \end{itemize}
\end{frame}

\begin{frame}[label=current]
    \frametitle{Planner's Objective}
    \framesubtitle{}
    \begin{itemize}
        \item In this research, the planner's objective function is the aggregate group outcome.
            Theorems and corollaries are based on it.
        \item If the planer's objective is to maximize welfare
            $W^*(\bm{\Sigma})=\sum_{i=1}^nu_i(\bm{x}^*(\bm{\Sigma}))=\frac{\beta+\gamma}{2\sum_{i=1}^n
            x_i^*(\bm{\Sigma})^2}$, the result of the key player is also possible in this case.
    \end{itemize}
\end{frame}

\begin{frame}[label=current]
    \frametitle{Group Targets}
    \framesubtitle{}
    \begin{itemize}
        \item This research characterizes a single-player target, but the idea of
            intercentrality measure can be generalized to a group index.
        \item The group target selection problem is not amenable to a sequential key
            player problem. In fact, optimal group targets belong to the maximization of
            submodular set functions, which cannot admit exact solutions.
    \end{itemize}
\end{frame}

\begin{frame}[label=current]
    \frametitle{Staged Games}
    \framesubtitle{}
    \begin{itemize}
        \item This method can be extended to solve a two-stage game.
            \begin{itemize}
                \item In the first stage, players decide simultaneously to stay in the network
                    $\eug$ or to drop out of it, then get their outside options and utilities.
                \item In the second stage, the staying players play the network game on the 
                    resulting network.
                \item A fun fact is that 
                    the authors themselves had solved the uniqueness of the second-stage
                    \Ne and the closed-form expression in \citet{author1} and \citet{author2}.
            \end{itemize}
    \end{itemize}
\end{frame}

\begin{frame}[allowframebreaks]
    \bibliography{ref}
\end{frame}
%\end{CJK*}
\end{document}

